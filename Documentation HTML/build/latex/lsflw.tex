% Generated by Sphinx.
\def\sphinxdocclass{report}
\documentclass[letterpaper,10pt,english]{sphinxmanual}
\usepackage[utf8]{inputenc}
\DeclareUnicodeCharacter{00A0}{\nobreakspace}
\usepackage{cmap}
\usepackage[T1]{fontenc}
\usepackage{babel}
\usepackage{times}
\usepackage[Bjarne]{fncychap}
\usepackage{longtable}
\usepackage{sphinx}
\usepackage{multirow}


\title{lsflw Documentation}
\date{January 25, 2015}
\release{1.0}
\author{CARON Cyril HOULEGATTE Pauline HOWELL Tuesday PINEAU Maxime}
\newcommand{\sphinxlogo}{}
\renewcommand{\releasename}{Release}
\makeindex

\makeatletter
\def\PYG@reset{\let\PYG@it=\relax \let\PYG@bf=\relax%
    \let\PYG@ul=\relax \let\PYG@tc=\relax%
    \let\PYG@bc=\relax \let\PYG@ff=\relax}
\def\PYG@tok#1{\csname PYG@tok@#1\endcsname}
\def\PYG@toks#1+{\ifx\relax#1\empty\else%
    \PYG@tok{#1}\expandafter\PYG@toks\fi}
\def\PYG@do#1{\PYG@bc{\PYG@tc{\PYG@ul{%
    \PYG@it{\PYG@bf{\PYG@ff{#1}}}}}}}
\def\PYG#1#2{\PYG@reset\PYG@toks#1+\relax+\PYG@do{#2}}

\expandafter\def\csname PYG@tok@vi\endcsname{\def\PYG@tc##1{\textcolor[rgb]{0.73,0.38,0.84}{##1}}}
\expandafter\def\csname PYG@tok@se\endcsname{\let\PYG@bf=\textbf\def\PYG@tc##1{\textcolor[rgb]{0.25,0.44,0.63}{##1}}}
\expandafter\def\csname PYG@tok@s\endcsname{\def\PYG@tc##1{\textcolor[rgb]{0.25,0.44,0.63}{##1}}}
\expandafter\def\csname PYG@tok@mb\endcsname{\def\PYG@tc##1{\textcolor[rgb]{0.13,0.50,0.31}{##1}}}
\expandafter\def\csname PYG@tok@gt\endcsname{\def\PYG@tc##1{\textcolor[rgb]{0.00,0.27,0.87}{##1}}}
\expandafter\def\csname PYG@tok@gs\endcsname{\let\PYG@bf=\textbf}
\expandafter\def\csname PYG@tok@w\endcsname{\def\PYG@tc##1{\textcolor[rgb]{0.73,0.73,0.73}{##1}}}
\expandafter\def\csname PYG@tok@kp\endcsname{\def\PYG@tc##1{\textcolor[rgb]{0.00,0.44,0.13}{##1}}}
\expandafter\def\csname PYG@tok@sc\endcsname{\def\PYG@tc##1{\textcolor[rgb]{0.25,0.44,0.63}{##1}}}
\expandafter\def\csname PYG@tok@sb\endcsname{\def\PYG@tc##1{\textcolor[rgb]{0.25,0.44,0.63}{##1}}}
\expandafter\def\csname PYG@tok@nf\endcsname{\def\PYG@tc##1{\textcolor[rgb]{0.02,0.16,0.49}{##1}}}
\expandafter\def\csname PYG@tok@kt\endcsname{\def\PYG@tc##1{\textcolor[rgb]{0.56,0.13,0.00}{##1}}}
\expandafter\def\csname PYG@tok@err\endcsname{\def\PYG@bc##1{\setlength{\fboxsep}{0pt}\fcolorbox[rgb]{1.00,0.00,0.00}{1,1,1}{\strut ##1}}}
\expandafter\def\csname PYG@tok@sr\endcsname{\def\PYG@tc##1{\textcolor[rgb]{0.14,0.33,0.53}{##1}}}
\expandafter\def\csname PYG@tok@o\endcsname{\def\PYG@tc##1{\textcolor[rgb]{0.40,0.40,0.40}{##1}}}
\expandafter\def\csname PYG@tok@go\endcsname{\def\PYG@tc##1{\textcolor[rgb]{0.20,0.20,0.20}{##1}}}
\expandafter\def\csname PYG@tok@gh\endcsname{\let\PYG@bf=\textbf\def\PYG@tc##1{\textcolor[rgb]{0.00,0.00,0.50}{##1}}}
\expandafter\def\csname PYG@tok@gp\endcsname{\let\PYG@bf=\textbf\def\PYG@tc##1{\textcolor[rgb]{0.78,0.36,0.04}{##1}}}
\expandafter\def\csname PYG@tok@nl\endcsname{\let\PYG@bf=\textbf\def\PYG@tc##1{\textcolor[rgb]{0.00,0.13,0.44}{##1}}}
\expandafter\def\csname PYG@tok@s1\endcsname{\def\PYG@tc##1{\textcolor[rgb]{0.25,0.44,0.63}{##1}}}
\expandafter\def\csname PYG@tok@nc\endcsname{\let\PYG@bf=\textbf\def\PYG@tc##1{\textcolor[rgb]{0.05,0.52,0.71}{##1}}}
\expandafter\def\csname PYG@tok@m\endcsname{\def\PYG@tc##1{\textcolor[rgb]{0.13,0.50,0.31}{##1}}}
\expandafter\def\csname PYG@tok@s2\endcsname{\def\PYG@tc##1{\textcolor[rgb]{0.25,0.44,0.63}{##1}}}
\expandafter\def\csname PYG@tok@mi\endcsname{\def\PYG@tc##1{\textcolor[rgb]{0.13,0.50,0.31}{##1}}}
\expandafter\def\csname PYG@tok@ni\endcsname{\let\PYG@bf=\textbf\def\PYG@tc##1{\textcolor[rgb]{0.84,0.33,0.22}{##1}}}
\expandafter\def\csname PYG@tok@sh\endcsname{\def\PYG@tc##1{\textcolor[rgb]{0.25,0.44,0.63}{##1}}}
\expandafter\def\csname PYG@tok@mf\endcsname{\def\PYG@tc##1{\textcolor[rgb]{0.13,0.50,0.31}{##1}}}
\expandafter\def\csname PYG@tok@no\endcsname{\def\PYG@tc##1{\textcolor[rgb]{0.38,0.68,0.84}{##1}}}
\expandafter\def\csname PYG@tok@kn\endcsname{\let\PYG@bf=\textbf\def\PYG@tc##1{\textcolor[rgb]{0.00,0.44,0.13}{##1}}}
\expandafter\def\csname PYG@tok@gu\endcsname{\let\PYG@bf=\textbf\def\PYG@tc##1{\textcolor[rgb]{0.50,0.00,0.50}{##1}}}
\expandafter\def\csname PYG@tok@gi\endcsname{\def\PYG@tc##1{\textcolor[rgb]{0.00,0.63,0.00}{##1}}}
\expandafter\def\csname PYG@tok@nn\endcsname{\let\PYG@bf=\textbf\def\PYG@tc##1{\textcolor[rgb]{0.05,0.52,0.71}{##1}}}
\expandafter\def\csname PYG@tok@c1\endcsname{\let\PYG@it=\textit\def\PYG@tc##1{\textcolor[rgb]{0.25,0.50,0.56}{##1}}}
\expandafter\def\csname PYG@tok@na\endcsname{\def\PYG@tc##1{\textcolor[rgb]{0.25,0.44,0.63}{##1}}}
\expandafter\def\csname PYG@tok@ne\endcsname{\def\PYG@tc##1{\textcolor[rgb]{0.00,0.44,0.13}{##1}}}
\expandafter\def\csname PYG@tok@gd\endcsname{\def\PYG@tc##1{\textcolor[rgb]{0.63,0.00,0.00}{##1}}}
\expandafter\def\csname PYG@tok@nv\endcsname{\def\PYG@tc##1{\textcolor[rgb]{0.73,0.38,0.84}{##1}}}
\expandafter\def\csname PYG@tok@kd\endcsname{\let\PYG@bf=\textbf\def\PYG@tc##1{\textcolor[rgb]{0.00,0.44,0.13}{##1}}}
\expandafter\def\csname PYG@tok@mo\endcsname{\def\PYG@tc##1{\textcolor[rgb]{0.13,0.50,0.31}{##1}}}
\expandafter\def\csname PYG@tok@mh\endcsname{\def\PYG@tc##1{\textcolor[rgb]{0.13,0.50,0.31}{##1}}}
\expandafter\def\csname PYG@tok@si\endcsname{\let\PYG@it=\textit\def\PYG@tc##1{\textcolor[rgb]{0.44,0.63,0.82}{##1}}}
\expandafter\def\csname PYG@tok@nd\endcsname{\let\PYG@bf=\textbf\def\PYG@tc##1{\textcolor[rgb]{0.33,0.33,0.33}{##1}}}
\expandafter\def\csname PYG@tok@sx\endcsname{\def\PYG@tc##1{\textcolor[rgb]{0.78,0.36,0.04}{##1}}}
\expandafter\def\csname PYG@tok@cm\endcsname{\let\PYG@it=\textit\def\PYG@tc##1{\textcolor[rgb]{0.25,0.50,0.56}{##1}}}
\expandafter\def\csname PYG@tok@vc\endcsname{\def\PYG@tc##1{\textcolor[rgb]{0.73,0.38,0.84}{##1}}}
\expandafter\def\csname PYG@tok@c\endcsname{\let\PYG@it=\textit\def\PYG@tc##1{\textcolor[rgb]{0.25,0.50,0.56}{##1}}}
\expandafter\def\csname PYG@tok@kr\endcsname{\let\PYG@bf=\textbf\def\PYG@tc##1{\textcolor[rgb]{0.00,0.44,0.13}{##1}}}
\expandafter\def\csname PYG@tok@nb\endcsname{\def\PYG@tc##1{\textcolor[rgb]{0.00,0.44,0.13}{##1}}}
\expandafter\def\csname PYG@tok@nt\endcsname{\let\PYG@bf=\textbf\def\PYG@tc##1{\textcolor[rgb]{0.02,0.16,0.45}{##1}}}
\expandafter\def\csname PYG@tok@vg\endcsname{\def\PYG@tc##1{\textcolor[rgb]{0.73,0.38,0.84}{##1}}}
\expandafter\def\csname PYG@tok@ss\endcsname{\def\PYG@tc##1{\textcolor[rgb]{0.32,0.47,0.09}{##1}}}
\expandafter\def\csname PYG@tok@ow\endcsname{\let\PYG@bf=\textbf\def\PYG@tc##1{\textcolor[rgb]{0.00,0.44,0.13}{##1}}}
\expandafter\def\csname PYG@tok@gr\endcsname{\def\PYG@tc##1{\textcolor[rgb]{1.00,0.00,0.00}{##1}}}
\expandafter\def\csname PYG@tok@cp\endcsname{\def\PYG@tc##1{\textcolor[rgb]{0.00,0.44,0.13}{##1}}}
\expandafter\def\csname PYG@tok@ge\endcsname{\let\PYG@it=\textit}
\expandafter\def\csname PYG@tok@sd\endcsname{\let\PYG@it=\textit\def\PYG@tc##1{\textcolor[rgb]{0.25,0.44,0.63}{##1}}}
\expandafter\def\csname PYG@tok@il\endcsname{\def\PYG@tc##1{\textcolor[rgb]{0.13,0.50,0.31}{##1}}}
\expandafter\def\csname PYG@tok@kc\endcsname{\let\PYG@bf=\textbf\def\PYG@tc##1{\textcolor[rgb]{0.00,0.44,0.13}{##1}}}
\expandafter\def\csname PYG@tok@bp\endcsname{\def\PYG@tc##1{\textcolor[rgb]{0.00,0.44,0.13}{##1}}}
\expandafter\def\csname PYG@tok@k\endcsname{\let\PYG@bf=\textbf\def\PYG@tc##1{\textcolor[rgb]{0.00,0.44,0.13}{##1}}}
\expandafter\def\csname PYG@tok@cs\endcsname{\def\PYG@tc##1{\textcolor[rgb]{0.25,0.50,0.56}{##1}}\def\PYG@bc##1{\setlength{\fboxsep}{0pt}\colorbox[rgb]{1.00,0.94,0.94}{\strut ##1}}}

\def\PYGZbs{\char`\\}
\def\PYGZus{\char`\_}
\def\PYGZob{\char`\{}
\def\PYGZcb{\char`\}}
\def\PYGZca{\char`\^}
\def\PYGZam{\char`\&}
\def\PYGZlt{\char`\<}
\def\PYGZgt{\char`\>}
\def\PYGZsh{\char`\#}
\def\PYGZpc{\char`\%}
\def\PYGZdl{\char`\$}
\def\PYGZhy{\char`\-}
\def\PYGZsq{\char`\'}
\def\PYGZdq{\char`\"}
\def\PYGZti{\char`\~}
% for compatibility with earlier versions
\def\PYGZat{@}
\def\PYGZlb{[}
\def\PYGZrb{]}
\makeatother

\renewcommand\PYGZsq{\textquotesingle}

\begin{document}

\maketitle
\tableofcontents
\phantomsection\label{index::doc}


Voici la documentation des classes pour la réalisation du mini projet


\chapter{Cellule}
\label{index:module-Cellule}\label{index:bienvenue-sur-cette-documentation}\label{index:cellule}\index{Cellule (module)}\index{Cellule (class in Cellule)}

\begin{fulllineitems}
\phantomsection\label{index:Cellule.Cellule}\pysiglinewithargsret{\strong{class }\code{Cellule.}\bfcode{Cellule}}{\emph{numero}, \emph{attaque}, \emph{defense}, \emph{attaqueMax}, \emph{defenseMax}, \emph{production}, \emph{couleur}, \emph{x}, \emph{y}, \emph{rayon}}{}
Une cellule du jeu. Elle peut être représentée par un sommet dans un graphe.
\begin{quote}\begin{description}
\item[{Parameters}] \leavevmode\begin{itemize}
\item {} 
\textbf{numero} (\href{http://docs.python.org/library/functions.html\#int}{\emph{int}}) -- le numéro identifiant la cellule (unique pour chaque cellule)

\item {} 
\textbf{attaque} (\href{http://docs.python.org/library/functions.html\#int}{\emph{int}}) -- le nombre d'unités attaquantes sur la cellule actuellement

\item {} 
\textbf{defense} (\href{http://docs.python.org/library/functions.html\#int}{\emph{int}}) -- le nombre d'unités defensives sur la cellule actuellement

\item {} 
\textbf{attaqueMax} (\href{http://docs.python.org/library/functions.html\#int}{\emph{int}}) -- le nombre d'unités attaquantes maximal sur la cellule

\item {} 
\textbf{defenseMax} (\href{http://docs.python.org/library/functions.html\#int}{\emph{int}}) -- le nombre d'unités defensives maximal sur la cellule

\item {} 
\textbf{production} (\href{http://docs.python.org/library/functions.html\#int}{\emph{int}}) -- vitesse de production des unités de la cellule

\item {} 
\textbf{couleurJoueur} (\href{http://docs.python.org/library/functions.html\#int}{\emph{int}}) -- couleur du joueur à qui appartient la cellule (-1 si neutre)

\item {} 
\textbf{x} (\href{http://docs.python.org/library/functions.html\#int}{\emph{int}}) -- la coordonnée x de la cellule sur le terrain (graphique)

\item {} 
\textbf{y} (\href{http://docs.python.org/library/functions.html\#int}{\emph{int}}) -- la coordonnée y de la cellule sur le terrain (graphique)

\item {} 
\textbf{rayon} (\href{http://docs.python.org/library/functions.html\#int}{\emph{int}}) -- le rayon de la cellule (graphique)

\end{itemize}

\item[{Raises}] \leavevmode\begin{itemize}
\item {} 
\textbf{CelluleException} -- si l'attaque est inférieure à 0 ou supérieure à l'attaque maximale

\item {} 
\textbf{CelluleException} -- si la défense est inférieure à 0 ou supérieure à la défense maximale

\end{itemize}

\end{description}\end{quote}
\index{aPourCouleur() (Cellule.Cellule method)}

\begin{fulllineitems}
\phantomsection\label{index:Cellule.Cellule.aPourCouleur}\pysiglinewithargsret{\bfcode{aPourCouleur}}{\emph{couleur}}{}
Retourne vrai si la cellule possède la couleur passée en paramètre, et faux sinon.
Autrement dit, elle retourne vrai si la cellule appartient au joueur ayant la couleur passée en paramètre.
\begin{quote}\begin{description}
\item[{Parameters}] \leavevmode
\textbf{couleur} (\href{http://docs.python.org/library/functions.html\#int}{\emph{int}}) -- la couleur à vérifier

\item[{Returns}] \leavevmode
retourne vrai si la cellule a bien pour couleur la couleur donnée en paramètre, faux sinon.

\item[{Return type}] \leavevmode
boolean

\end{description}\end{quote}

\end{fulllineitems}

\index{ajouterLien() (Cellule.Cellule method)}

\begin{fulllineitems}
\phantomsection\label{index:Cellule.Cellule.ajouterLien}\pysiglinewithargsret{\bfcode{ajouterLien}}{\emph{lien}}{}
Ajoute un lien reliant cette cellule à une autre.
\begin{quote}\begin{description}
\item[{Parameters}] \leavevmode
\textbf{lien} ({\hyperref[index:module-Lien]{\emph{Lien}}}) -- le lien à ajouter

\item[{Raises CelluleException}] \leavevmode
si cette cellule n'est pas l'une des cellules aux extrémités du lien.

\end{description}\end{quote}

\end{fulllineitems}

\index{getAttaque() (Cellule.Cellule method)}

\begin{fulllineitems}
\phantomsection\label{index:Cellule.Cellule.getAttaque}\pysiglinewithargsret{\bfcode{getAttaque}}{}{}
Retourne l'attaque actuelle de la cellule.
\begin{quote}\begin{description}
\item[{Returns}] \leavevmode
l'attaque de la cellule

\item[{Return type}] \leavevmode
int

\end{description}\end{quote}

\end{fulllineitems}

\index{getAttaqueMax() (Cellule.Cellule method)}

\begin{fulllineitems}
\phantomsection\label{index:Cellule.Cellule.getAttaqueMax}\pysiglinewithargsret{\bfcode{getAttaqueMax}}{}{}
Retourne la quantité maximale d'unités offensives que la cellule peut contenir
\begin{quote}\begin{description}
\item[{Returns}] \leavevmode
l'attaque maximale de la cellule

\item[{Return type}] \leavevmode
int

\end{description}\end{quote}

\end{fulllineitems}

\index{getCouleur() (Cellule.Cellule method)}

\begin{fulllineitems}
\phantomsection\label{index:Cellule.Cellule.getCouleur}\pysiglinewithargsret{\bfcode{getCouleur}}{}{}
Retourne la couleur de la cellule. Cette couleur correspond à un numéro de joueur supérieur ou égale à -1 (-1 étant la couleur du neutre).
Cette couleur permet de définir à quel joueur appartient la cellule.
\begin{quote}\begin{description}
\item[{Returns}] \leavevmode
la couleur de la cellule

\item[{Return type}] \leavevmode
int

\end{description}\end{quote}

\end{fulllineitems}

\index{getCout() (Cellule.Cellule method)}

\begin{fulllineitems}
\phantomsection\label{index:Cellule.Cellule.getCout}\pysiglinewithargsret{\bfcode{getCout}}{}{}
Retourne le coût de base de la cellule pour être conquise, c'est à dire le nombre d'unités nécéssaire pour la capturer ( attaque + défense ).
On ne prend pas en compte les unités présentes sur les liens.
\begin{quote}\begin{description}
\item[{Returns}] \leavevmode
le coût de la cellule

\item[{Return type}] \leavevmode
int

\end{description}\end{quote}

\end{fulllineitems}

\index{getDefense() (Cellule.Cellule method)}

\begin{fulllineitems}
\phantomsection\label{index:Cellule.Cellule.getDefense}\pysiglinewithargsret{\bfcode{getDefense}}{}{}
Retourne la défense actuelle de la cellule.
\begin{quote}\begin{description}
\item[{Returns}] \leavevmode
la défense de la cellule

\item[{Return type}] \leavevmode
int

\end{description}\end{quote}

\end{fulllineitems}

\index{getDefenseMax() (Cellule.Cellule method)}

\begin{fulllineitems}
\phantomsection\label{index:Cellule.Cellule.getDefenseMax}\pysiglinewithargsret{\bfcode{getDefenseMax}}{}{}
Retourne la quantité maximale d'unités défensives que la cellule peut contenir
\begin{quote}\begin{description}
\item[{Returns}] \leavevmode
la défense maximale de la cellule

\item[{Return type}] \leavevmode
int

\end{description}\end{quote}

\end{fulllineitems}

\index{getExcedent() (Cellule.Cellule method)}

\begin{fulllineitems}
\phantomsection\label{index:Cellule.Cellule.getExcedent}\pysiglinewithargsret{\bfcode{getExcedent}}{}{}
Retourne l'excédent de la cellule. 
L'excédent correspond au trop plein d'unités que la cellule va produire ou recevoir, et perdre (car elle aura déjà atteint son attaque maximale).
Seules les unités des mouvements arrivant vers cette cellule avec un temps restant inférieur à 1000 sont prises en comptes.
\begin{quote}\begin{description}
\item[{Returns}] \leavevmode
le nombre d'unités excédent

\item[{Return type}] \leavevmode
int

\end{description}\end{quote}

\end{fulllineitems}

\index{getLiens() (Cellule.Cellule method)}

\begin{fulllineitems}
\phantomsection\label{index:Cellule.Cellule.getLiens}\pysiglinewithargsret{\bfcode{getLiens}}{}{}
Retourne les liens dont cette cellule est l'une des extrémités.
\begin{quote}\begin{description}
\item[{Returns}] \leavevmode
la liste des liens

\item[{Return type}] \leavevmode
List\textless{}Lien\textgreater{}

\end{description}\end{quote}

\end{fulllineitems}

\index{getMouvementsVersCellule() (Cellule.Cellule method)}

\begin{fulllineitems}
\phantomsection\label{index:Cellule.Cellule.getMouvementsVersCellule}\pysiglinewithargsret{\bfcode{getMouvementsVersCellule}}{}{}
Méthode qui retourne la liste de tous les mouvements allant vers cette cellule.
\begin{quote}\begin{description}
\item[{Returns}] \leavevmode
la liste de tous les mouvements allant vers cette cellule.

\item[{Return type}] \leavevmode
List\textless{}Mouvement\textgreater{}

\end{description}\end{quote}

\end{fulllineitems}

\index{getNumero() (Cellule.Cellule method)}

\begin{fulllineitems}
\phantomsection\label{index:Cellule.Cellule.getNumero}\pysiglinewithargsret{\bfcode{getNumero}}{}{}
Retourne le numéro unique identifiant la cellule
\begin{quote}\begin{description}
\item[{Returns}] \leavevmode
le numéro de la cellule

\item[{Return type}] \leavevmode
int

\end{description}\end{quote}

\end{fulllineitems}

\index{getPourcentageAttaque() (Cellule.Cellule method)}

\begin{fulllineitems}
\phantomsection\label{index:Cellule.Cellule.getPourcentageAttaque}\pysiglinewithargsret{\bfcode{getPourcentageAttaque}}{}{}
Retourne à combien de pourcentage la cellule est pleine.
1 si l'attaque de la cellule est égale à l'attaque maximale.
\begin{quote}\begin{description}
\item[{Returns}] \leavevmode
le pourcentage de l'attaque de la cellule

\item[{Return type}] \leavevmode
int

\end{description}\end{quote}

\end{fulllineitems}

\index{getProduction() (Cellule.Cellule method)}

\begin{fulllineitems}
\phantomsection\label{index:Cellule.Cellule.getProduction}\pysiglinewithargsret{\bfcode{getProduction}}{}{}
Retourne la valeur de la cadence de production de la cellule
\begin{quote}\begin{description}
\item[{Returns}] \leavevmode
la cadence de production de la cellule

\item[{Return type}] \leavevmode
int

\end{description}\end{quote}

\end{fulllineitems}

\index{getVoisins() (Cellule.Cellule method)}

\begin{fulllineitems}
\phantomsection\label{index:Cellule.Cellule.getVoisins}\pysiglinewithargsret{\bfcode{getVoisins}}{}{}
Retourne la liste des cellules voisines à celle-ci.

ATTENTION, peut renvoyer un résultat différent de la méthode Terrain.getVoisinsCellule( cellule )
car on ne modifiera jamais la liste des liens de la cellule, 
contrairement au terrain, ou l'on pourra ne considérer qu'une partie du terrain (un sous graphe), et donc qu'une partie des liens.
\begin{quote}\begin{description}
\item[{Returns}] \leavevmode
la liste des cellules voisines

\item[{Return type}] \leavevmode
List\textless{}Cellule\textgreater{}

\end{description}\end{quote}

\end{fulllineitems}

\index{getVoisinsAllies() (Cellule.Cellule method)}

\begin{fulllineitems}
\phantomsection\label{index:Cellule.Cellule.getVoisinsAllies}\pysiglinewithargsret{\bfcode{getVoisinsAllies}}{}{}
Retourne la liste des cellules alliées de cette cellule, c'est à dire celles qui ont la même couleur que celle de cette cellule.
\begin{quote}\begin{description}
\item[{Returns}] \leavevmode
la liste des cellules alliées de cette cellule

\item[{Return type}] \leavevmode
List\textless{}Cellule\textgreater{}

\end{description}\end{quote}

\end{fulllineitems}

\index{getVoisinsEnnemis() (Cellule.Cellule method)}

\begin{fulllineitems}
\phantomsection\label{index:Cellule.Cellule.getVoisinsEnnemis}\pysiglinewithargsret{\bfcode{getVoisinsEnnemis}}{}{}
Retourne la liste des cellules ennemies de cette cellule, c'est à dire celles qui n'ont pas la même couleur que celle de cette cellule.
\begin{quote}\begin{description}
\item[{Returns}] \leavevmode
la liste des cellules ennemies de cette cellule

\item[{Return type}] \leavevmode
List\textless{}Cellule\textgreater{}

\end{description}\end{quote}

\end{fulllineitems}

\index{setAttaque() (Cellule.Cellule method)}

\begin{fulllineitems}
\phantomsection\label{index:Cellule.Cellule.setAttaque}\pysiglinewithargsret{\bfcode{setAttaque}}{\emph{attaque}}{}
Affecte la valeur de la variable attaque actuelle.
La nouvelle valeur de l'attaque ne peut pas être inférieure à 0, ni excéder l'attaque maximale de la cellule.
\begin{quote}\begin{description}
\item[{Parameters}] \leavevmode
\textbf{attaque} (\href{http://docs.python.org/library/functions.html\#int}{\emph{int}}) -- la nouvelle valeur de l'attaque actuelle de la cellule

\item[{Raises CelluleException}] \leavevmode
si l'attaque entrée en paramètre est inférieure à 0 ou supérieure à l'attaque maximale

\end{description}\end{quote}

\end{fulllineitems}

\index{setCouleur() (Cellule.Cellule method)}

\begin{fulllineitems}
\phantomsection\label{index:Cellule.Cellule.setCouleur}\pysiglinewithargsret{\bfcode{setCouleur}}{\emph{couleur}}{}
Affecte la valeur de la variable couleur (la cellule change de propriétaire). 
Cette couleur doit être supérieure ou égale à -1, -1 étant la couleur du joueur neutre.
\begin{quote}\begin{description}
\item[{Parameters}] \leavevmode
\textbf{couleur} (\href{http://docs.python.org/library/functions.html\#int}{\emph{int}}) -- la nouvelle couleur de la cellule.

\item[{Raises CelluleException}] \leavevmode
si la couleur n'est pas supérieure ou égale à -1.

\end{description}\end{quote}

\end{fulllineitems}

\index{setDefense() (Cellule.Cellule method)}

\begin{fulllineitems}
\phantomsection\label{index:Cellule.Cellule.setDefense}\pysiglinewithargsret{\bfcode{setDefense}}{\emph{defense}}{}
Affecte la valeur de la variable defense actuelle.
La nouvelle valeur de la défense ne peut pas être inférieure à 0, ni excéder la défense maximale de la cellule.
\begin{quote}\begin{description}
\item[{Parameters}] \leavevmode
\textbf{defense} (\href{http://docs.python.org/library/functions.html\#int}{\emph{int}}) -- la nouvelle valeur de la défense actuelle de la cellule

\item[{Raises CelluleException}] \leavevmode
si la défense entrée en paramètre est inférieure à 0 ou supérieure à la défense maximale

\end{description}\end{quote}

\end{fulllineitems}

\index{toString() (Cellule.Cellule method)}

\begin{fulllineitems}
\phantomsection\label{index:Cellule.Cellule.toString}\pysiglinewithargsret{\bfcode{toString}}{}{}
Représente l'objet sous forme de chaine
\begin{quote}\begin{description}
\item[{Returns}] \leavevmode
retourne une chaine de caractère représentant la cellule

\item[{Return type}] \leavevmode
str

\end{description}\end{quote}

\end{fulllineitems}

\index{vaEtrePrise() (Cellule.Cellule method)}

\begin{fulllineitems}
\phantomsection\label{index:Cellule.Cellule.vaEtrePrise}\pysiglinewithargsret{\bfcode{vaEtrePrise}}{}{}
vérifie si la cellule va être conquise par un autre joueur (elle va changer de propriétaire).
Si c'est le cas, elle retourne True, False sinon.
\begin{quote}\begin{description}
\item[{Returns}] \leavevmode
retourne vrai si cette cellule va être conquise par un autre joueur, faux sinon

\item[{Return type}] \leavevmode
boolean

\end{description}\end{quote}

\end{fulllineitems}


\end{fulllineitems}



\chapter{Lien}
\label{index:module-Lien}\label{index:lien}\index{Lien (module)}\index{Lien (class in Lien)}

\begin{fulllineitems}
\phantomsection\label{index:Lien.Lien}\pysiglinewithargsret{\strong{class }\code{Lien.}\bfcode{Lien}}{\emph{u}, \emph{v}, \emph{distance}}{}
Les liens reliant les cellules du terrain (représentent les arêtes du graphe).

On mettra toujours la cellule ayant le plus petit numéro dans l'attribut u.
\begin{quote}\begin{description}
\item[{Parameters}] \leavevmode\begin{itemize}
\item {} 
\textbf{u} ({\hyperref[index:module-Cellule]{\emph{Cellule}}}) -- Une cellule de l'une des extrémités du lien

\item {} 
\textbf{v} ({\hyperref[index:module-Cellule]{\emph{Cellule}}}) -- L'autre cellule de l'une des extrémités du lien

\item {} 
\textbf{distance} (\href{http://docs.python.org/library/functions.html\#int}{\emph{int}}) -- la distance séparant les cellules u et v

\end{itemize}

\item[{Raises LienException}] \leavevmode
si la distance est inférieure ou égale à 0

\end{description}\end{quote}
\index{ajouterMouvementVersCellule() (Lien.Lien method)}

\begin{fulllineitems}
\phantomsection\label{index:Lien.Lien.ajouterMouvementVersCellule}\pysiglinewithargsret{\bfcode{ajouterMouvementVersCellule}}{\emph{cellule}, \emph{mouvement}}{}
Ajoute le mouvement passé en paramètre vers la cellule passée en paramètre
\begin{quote}\begin{description}
\item[{Parameters}] \leavevmode\begin{itemize}
\item {} 
\textbf{cellule} ({\hyperref[index:module-Cellule]{\emph{Cellule}}}) -- La cellule à laquelle on ajoute le mouvement

\item {} 
\textbf{mouvement} ({\hyperref[index:module-Mouvement]{\emph{Mouvement}}}) -- Le mouvement à ajouter

\end{itemize}

\item[{Raises LienException}] \leavevmode
si la cellule n'appartient pas au lien

\end{description}\end{quote}

\end{fulllineitems}

\index{ajouterMouvementVersU() (Lien.Lien method)}

\begin{fulllineitems}
\phantomsection\label{index:Lien.Lien.ajouterMouvementVersU}\pysiglinewithargsret{\bfcode{ajouterMouvementVersU}}{\emph{mouvement}}{}
Ajoute un mouvement vers la cellule enregistrée sous U
\begin{quote}\begin{description}
\item[{Parameters}] \leavevmode
\textbf{mouvement} ({\hyperref[index:module-Mouvement]{\emph{Mouvement}}}) -- Le mouvement à ajouter

\end{description}\end{quote}

\end{fulllineitems}

\index{ajouterMouvementVersV() (Lien.Lien method)}

\begin{fulllineitems}
\phantomsection\label{index:Lien.Lien.ajouterMouvementVersV}\pysiglinewithargsret{\bfcode{ajouterMouvementVersV}}{\emph{mouvement}}{}
Ajoute un mouvement vers la cellule enregistrée sous V
\begin{quote}\begin{description}
\item[{Parameters}] \leavevmode
\textbf{mouvement} ({\hyperref[index:module-Mouvement]{\emph{Mouvement}}}) -- Le mouvement à ajouter.

\end{description}\end{quote}

\end{fulllineitems}

\index{celluleAppartientAuLien() (Lien.Lien method)}

\begin{fulllineitems}
\phantomsection\label{index:Lien.Lien.celluleAppartientAuLien}\pysiglinewithargsret{\bfcode{celluleAppartientAuLien}}{\emph{cellule}}{}
Retourne vrai si la cellule passée en paramètre appartient au lien, faux sinon.
\begin{quote}\begin{description}
\item[{Parameters}] \leavevmode
\textbf{cellule} ({\hyperref[index:module-Cellule]{\emph{Cellule}}}) -- Cellule dont on veut savoir si elle appartient au lien

\item[{Return type}] \leavevmode
booleen

\end{description}\end{quote}

\end{fulllineitems}

\index{clearAllMouvements() (Lien.Lien method)}

\begin{fulllineitems}
\phantomsection\label{index:Lien.Lien.clearAllMouvements}\pysiglinewithargsret{\bfcode{clearAllMouvements}}{}{}
Supprime tous les mouvements présents sur le lien.

\end{fulllineitems}

\index{getDistance() (Lien.Lien method)}

\begin{fulllineitems}
\phantomsection\label{index:Lien.Lien.getDistance}\pysiglinewithargsret{\bfcode{getDistance}}{}{}
Retourne la longueur du lien, c'est à dire la distance séparant les deux cellules aux extrémités du lien.
\begin{quote}\begin{description}
\item[{Returns}] \leavevmode
la longueur du lien

\item[{Return type}] \leavevmode
int

\end{description}\end{quote}

\end{fulllineitems}

\index{getMouvementsVersCellule() (Lien.Lien method)}

\begin{fulllineitems}
\phantomsection\label{index:Lien.Lien.getMouvementsVersCellule}\pysiglinewithargsret{\bfcode{getMouvementsVersCellule}}{\emph{cellule}}{}
Retourne la liste des mouvements allant vers la cellule passée en paramètre.
\begin{quote}\begin{description}
\item[{Parameters}] \leavevmode
\textbf{cellule} ({\hyperref[index:module-Cellule]{\emph{Cellule}}}) -- La cellule dont on veut récupérer les mouvements entrants

\item[{Returns}] \leavevmode
la liste des mouvements allant vers cette cellule

\item[{Return type}] \leavevmode
List\textless{}Mouvement\textgreater{}

\item[{Raises LienException}] \leavevmode
si la cellule n'appartient pas au lien

\end{description}\end{quote}

\end{fulllineitems}

\index{getMouvementsVersU() (Lien.Lien method)}

\begin{fulllineitems}
\phantomsection\label{index:Lien.Lien.getMouvementsVersU}\pysiglinewithargsret{\bfcode{getMouvementsVersU}}{}{}
Retourne la liste des mouvements allant vers U
\begin{quote}\begin{description}
\item[{Returns}] \leavevmode
la liste des mouvements allant vers U

\item[{Return type}] \leavevmode
List\textless{}Mouvement\textgreater{}

\end{description}\end{quote}

\end{fulllineitems}

\index{getMouvementsVersV() (Lien.Lien method)}

\begin{fulllineitems}
\phantomsection\label{index:Lien.Lien.getMouvementsVersV}\pysiglinewithargsret{\bfcode{getMouvementsVersV}}{}{}
Retourne la liste des mouvements allant vers V
\begin{quote}\begin{description}
\item[{Returns}] \leavevmode
la liste des mouvements allant vers V

\item[{Return type}] \leavevmode
List\textless{}Mouvement\textgreater{}

\end{description}\end{quote}

\end{fulllineitems}

\index{getOtherCellule() (Lien.Lien method)}

\begin{fulllineitems}
\phantomsection\label{index:Lien.Lien.getOtherCellule}\pysiglinewithargsret{\bfcode{getOtherCellule}}{\emph{cellule\_inconnue}}{}
Selon la cellule donnée en paramètre, retourne l'autre cellule du lien (retourne U si on donne V et vice versa).
\begin{quote}\begin{description}
\item[{Raises LienException}] \leavevmode
si la cellule inconnue n'appartient pas au lien

\item[{Returns}] \leavevmode
l'autre cellule du lien

\item[{Return type}] \leavevmode
Cellule

\end{description}\end{quote}

\end{fulllineitems}

\index{getU() (Lien.Lien method)}

\begin{fulllineitems}
\phantomsection\label{index:Lien.Lien.getU}\pysiglinewithargsret{\bfcode{getU}}{}{}
Retourne la cellule enregistrée sous l'attribut U
\begin{quote}\begin{description}
\item[{Returns}] \leavevmode
la cellule enregistrée dans U

\item[{Return type}] \leavevmode
Cellule

\end{description}\end{quote}

\end{fulllineitems}

\index{getV() (Lien.Lien method)}

\begin{fulllineitems}
\phantomsection\label{index:Lien.Lien.getV}\pysiglinewithargsret{\bfcode{getV}}{}{}
Retourne la cellule enregistrée sous l'attribut V
\begin{quote}\begin{description}
\item[{Returns}] \leavevmode
la cellule enregistrée dans V

\item[{Return type}] \leavevmode
Cellule

\end{description}\end{quote}

\end{fulllineitems}

\index{hachage() (Lien.Lien method)}

\begin{fulllineitems}
\phantomsection\label{index:Lien.Lien.hachage}\pysiglinewithargsret{\bfcode{hachage}}{\emph{cellule1}, \emph{cellule2}}{}
Permet de calculer la valeur unique qui identifie un lien supposé entre deux cellules, 
Retourne la valeur du hash d'un lien à partir de ces deux cellules.
\begin{quote}\begin{description}
\item[{Parameters}] \leavevmode\begin{itemize}
\item {} 
\textbf{cellule1} ({\hyperref[index:module-Cellule]{\emph{Cellule}}}) -- la première cellule

\item {} 
\textbf{cellule2} ({\hyperref[index:module-Cellule]{\emph{Cellule}}}) -- la deuxième cellule

\end{itemize}

\item[{Returns}] \leavevmode
l'identifiant du lien supposé entre les deux cellules

\item[{Return type}] \leavevmode
int

\end{description}\end{quote}

\end{fulllineitems}

\index{hash() (Lien.Lien method)}

\begin{fulllineitems}
\phantomsection\label{index:Lien.Lien.hash}\pysiglinewithargsret{\bfcode{hash}}{}{}
Utilisée pour ranger les liens dans un dictionnaire
Retourne la valeur unique qui identifie le lien
\begin{quote}\begin{description}
\item[{Returns}] \leavevmode
la valeur unique identifiant le lien.

\item[{Return type}] \leavevmode
int

\end{description}\end{quote}

\end{fulllineitems}

\index{toString() (Lien.Lien method)}

\begin{fulllineitems}
\phantomsection\label{index:Lien.Lien.toString}\pysiglinewithargsret{\bfcode{toString}}{}{}
Retourne des informations textuelles sur le lien
\begin{quote}\begin{description}
\item[{Returns}] \leavevmode
le lien sous forme d'une chaine de caractère

\item[{Return type}] \leavevmode
str

\end{description}\end{quote}

\end{fulllineitems}


\end{fulllineitems}



\chapter{Mouvement}
\label{index:mouvement}\label{index:module-Mouvement}\index{Mouvement (module)}\index{Mouvement (class in Mouvement)}

\begin{fulllineitems}
\phantomsection\label{index:Mouvement.Mouvement}\pysiglinewithargsret{\strong{class }\code{Mouvement.}\bfcode{Mouvement}}{\emph{depuis}, \emph{vers}, \emph{nbUnites}, \emph{couleurJoueur}, \emph{distance}, \emph{vitesse}, \emph{temps\_depart}, \emph{temps\_actuel}}{}
Un mouvement représente un déplacement d'unités sur un lien.
\begin{quote}\begin{description}
\item[{Parameters}] \leavevmode\begin{itemize}
\item {} 
\textbf{depuis} ({\hyperref[index:module-Cellule]{\emph{Cellule}}}) -- la cellule au départ du mouvement

\item {} 
\textbf{vers} ({\hyperref[index:module-Cellule]{\emph{Cellule}}}) -- la cellule à l'arrivée du mouvement

\item {} 
\textbf{nbUnites} (\href{http://docs.python.org/library/functions.html\#int}{\emph{int}}) -- le nombre d'unités qui sont sur le mouvement

\item {} 
\textbf{couleurJoueur} (\href{http://docs.python.org/library/functions.html\#int}{\emph{int}}) -- le numéro du joueur auquel appartiennent les unités offensives

\item {} 
\textbf{distance} (\href{http://docs.python.org/library/functions.html\#int}{\emph{int}}) -- la distance à parcourir sur le lien

\item {} 
\textbf{temps\_depart} (\href{http://docs.python.org/library/functions.html\#int}{\emph{int}}) -- temps du serveur  lors de l'envoi du mouvement

\item {} 
\textbf{temps\_actuel} (\href{http://docs.python.org/library/functions.html\#int}{\emph{int}}) -- temps du serveur

\end{itemize}

\end{description}\end{quote}
\index{aPourCouleur() (Mouvement.Mouvement method)}

\begin{fulllineitems}
\phantomsection\label{index:Mouvement.Mouvement.aPourCouleur}\pysiglinewithargsret{\bfcode{aPourCouleur}}{\emph{couleurJoueur}}{}
Retourne vrai si le mouvement possède la couleur passée en paramètre.
(donc si le mouvement appartient au joueur ayant cette couleur)
\begin{quote}\begin{description}
\item[{Parameters}] \leavevmode
\textbf{couleurJoueur} (\href{http://docs.python.org/library/functions.html\#int}{\emph{int}}) -- la couleur du joueur

\item[{Returns}] \leavevmode
vrai si le mouvement possède cette couleur, faux sinon.

\item[{Return type}] \leavevmode
booleen

\end{description}\end{quote}

\end{fulllineitems}

\index{fromCellule() (Mouvement.Mouvement method)}

\begin{fulllineitems}
\phantomsection\label{index:Mouvement.Mouvement.fromCellule}\pysiglinewithargsret{\bfcode{fromCellule}}{}{}
Retourne la cellule depuis laquelle le mouvement est originaire
\begin{quote}\begin{description}
\item[{Returns}] \leavevmode
la cellule depuis laquelle le mouvement est originaire

\item[{Return type}] \leavevmode
Cellule

\end{description}\end{quote}

\end{fulllineitems}

\index{getCouleur() (Mouvement.Mouvement method)}

\begin{fulllineitems}
\phantomsection\label{index:Mouvement.Mouvement.getCouleur}\pysiglinewithargsret{\bfcode{getCouleur}}{}{}
Retourne la couleur du mouvement, c'est à dire le numéro du joueur qui envoie le mouvement.
\begin{quote}\begin{description}
\item[{Returns}] \leavevmode
la couleur du mouvement

\item[{Return type}] \leavevmode
int

\end{description}\end{quote}

\end{fulllineitems}

\index{getDistance() (Mouvement.Mouvement method)}

\begin{fulllineitems}
\phantomsection\label{index:Mouvement.Mouvement.getDistance}\pysiglinewithargsret{\bfcode{getDistance}}{}{}
Retourne la distance totale à parcourir par le mouvement.
\begin{quote}\begin{description}
\item[{Returns}] \leavevmode
la distance totale à parcourir par le mouvement

\item[{Return type}] \leavevmode
int

\end{description}\end{quote}

\end{fulllineitems}

\index{getNbUnites() (Mouvement.Mouvement method)}

\begin{fulllineitems}
\phantomsection\label{index:Mouvement.Mouvement.getNbUnites}\pysiglinewithargsret{\bfcode{getNbUnites}}{}{}
Retourne le nombre d'unités sur le mouvement
\begin{quote}\begin{description}
\item[{Returns}] \leavevmode
le nombre d'unités

\item[{Return type}] \leavevmode
int

\end{description}\end{quote}

\end{fulllineitems}

\index{getTempsActuel() (Mouvement.Mouvement method)}

\begin{fulllineitems}
\phantomsection\label{index:Mouvement.Mouvement.getTempsActuel}\pysiglinewithargsret{\bfcode{getTempsActuel}}{}{}
Retourne le temps actuel du serveur
\begin{quote}\begin{description}
\item[{Returns}] \leavevmode
le temps actuel du serveur

\item[{Return type}] \leavevmode
int

\end{description}\end{quote}

\end{fulllineitems}

\index{getTempsDepart() (Mouvement.Mouvement method)}

\begin{fulllineitems}
\phantomsection\label{index:Mouvement.Mouvement.getTempsDepart}\pysiglinewithargsret{\bfcode{getTempsDepart}}{}{}
Retourne le temps de départ du mouvement
\begin{quote}\begin{description}
\item[{Returns}] \leavevmode
le temps de départ du mouvement

\item[{Return type}] \leavevmode
int

\end{description}\end{quote}

\end{fulllineitems}

\index{getTempsRestant() (Mouvement.Mouvement method)}

\begin{fulllineitems}
\phantomsection\label{index:Mouvement.Mouvement.getTempsRestant}\pysiglinewithargsret{\bfcode{getTempsRestant}}{}{}
Retourne le temps restant à parcourir avant l'arrivée du mouvement à destination.
\begin{quote}\begin{description}
\item[{Returns}] \leavevmode
le temps restant à parcourir

\item[{Return type}] \leavevmode
int

\end{description}\end{quote}

\end{fulllineitems}

\index{getVitesse() (Mouvement.Mouvement method)}

\begin{fulllineitems}
\phantomsection\label{index:Mouvement.Mouvement.getVitesse}\pysiglinewithargsret{\bfcode{getVitesse}}{}{}
Retourne la vitesse du mouvement
\begin{quote}\begin{description}
\item[{Returns}] \leavevmode
la vitesse du mouvement

\item[{Return type}] \leavevmode
int

\end{description}\end{quote}

\end{fulllineitems}

\index{setTempsActuel() (Mouvement.Mouvement method)}

\begin{fulllineitems}
\phantomsection\label{index:Mouvement.Mouvement.setTempsActuel}\pysiglinewithargsret{\bfcode{setTempsActuel}}{\emph{temps\_actuel}}{}
Affecte la variable temps\_actuel avec la valeur passée en paramètre
\begin{quote}\begin{description}
\item[{Parameters}] \leavevmode
\textbf{temps\_actuel} (\href{http://docs.python.org/library/functions.html\#int}{\emph{int}}) -- le temps du serveur

\end{description}\end{quote}

\end{fulllineitems}

\index{toCellule() (Mouvement.Mouvement method)}

\begin{fulllineitems}
\phantomsection\label{index:Mouvement.Mouvement.toCellule}\pysiglinewithargsret{\bfcode{toCellule}}{}{}
Retourne la cellule vers laquelle le mouvement se dirige
\begin{quote}\begin{description}
\item[{Returns}] \leavevmode
la cellule vers laquelle le mouvement se dirige

\item[{Return type}] \leavevmode
Cellule

\end{description}\end{quote}

\end{fulllineitems}

\index{toOrder() (Mouvement.Mouvement method)}

\begin{fulllineitems}
\phantomsection\label{index:Mouvement.Mouvement.toOrder}\pysiglinewithargsret{\bfcode{toOrder}}{\emph{uid}}{}
Retourne l'ordre correspondant au mouvement associé dans la forme du protocole du serveur
\begin{quote}\begin{description}
\item[{Returns}] \leavevmode
l'ordre correspondant au mouvement

\item[{Return type}] \leavevmode
str

\end{description}\end{quote}

\end{fulllineitems}


\end{fulllineitems}



\chapter{Robot}
\label{index:module-Robot}\label{index:robot}\index{Robot (module)}\index{Robot (class in Robot)}

\begin{fulllineitems}
\phantomsection\label{index:Robot.Robot}\pysiglinewithargsret{\strong{class }\code{Robot.}\bfcode{Robot}}{\emph{uid}}{}
La classe Robot. 
Initialise le robot.
A appeler dans la procédure `register\_pooo(uid)'
\begin{quote}\begin{description}
\item[{Parameters}] \leavevmode
\textbf{uid} (\href{http://docs.python.org/library/functions.html\#str}{\emph{str}}) -- identifiant unique du robot que le serveur lui a attribué

\end{description}\end{quote}
\index{analyseMessage() (Robot.Robot method)}

\begin{fulllineitems}
\phantomsection\label{index:Robot.Robot.analyseMessage}\pysiglinewithargsret{\bfcode{analyseMessage}}{\emph{state}}{}~\begin{description}
\item[{Décompose la chaine state passée en paramètre et agit en conséquence, exécute soit :}] \leavevmode\begin{itemize}
\item {} 
state of game

\item {} 
game over

\item {} 
end of game

\end{itemize}

\end{description}
\begin{quote}\begin{description}
\item[{Parameters}] \leavevmode
\textbf{state} (\href{http://docs.python.org/library/functions.html\#str}{\emph{str}}) -- la chaine reçue, envoyée par le serveur, peut contenir un message STATE, GAMEOVER, ou ENDOFGAME

\end{description}\end{quote}

\end{fulllineitems}

\index{end\_of\_game() (Robot.Robot method)}

\begin{fulllineitems}
\phantomsection\label{index:Robot.Robot.end_of_game}\pysiglinewithargsret{\bfcode{end\_of\_game}}{}{}
Arrête le match en cours

\end{fulllineitems}

\index{game\_over() (Robot.Robot method)}

\begin{fulllineitems}
\phantomsection\label{index:Robot.Robot.game_over}\pysiglinewithargsret{\bfcode{game\_over}}{\emph{state\_game\_over}}{}
L'un des participants du match a perdu, on analyse plus en détail la chaine reçue pour savoir si c'est ce robot qui ne peut plus jouer.
\begin{quote}\begin{description}
\item[{Parameters}] \leavevmode
\textbf{state\_game\_over} (\href{http://docs.python.org/library/functions.html\#str}{\emph{str}}) -- chaine de caractère GAMEOVER envoyé par le serveur

\end{description}\end{quote}

\end{fulllineitems}

\index{getDecisions() (Robot.Robot method)}

\begin{fulllineitems}
\phantomsection\label{index:Robot.Robot.getDecisions}\pysiglinewithargsret{\bfcode{getDecisions}}{}{}
Retourne la liste des décisions, chacune conforme au protocole du serveur. 
Ces décisions sont prises selon la stratégie adoptée.

exemple : {[} `{[}0947e717-02a1-4d83-9470-a941b6e8ed07{]}100FROM0TO2', `{[}0947e717-02a1-4d83-9470-a941b6e8ed07{]}56FROM6TO4' {]}
\begin{quote}\begin{description}
\item[{Returns}] \leavevmode
la liste des décisions

\item[{Return type}] \leavevmode
List\textless{}str\textgreater{}

\end{description}\end{quote}

\end{fulllineitems}

\index{getIdMatch() (Robot.Robot method)}

\begin{fulllineitems}
\phantomsection\label{index:Robot.Robot.getIdMatch}\pysiglinewithargsret{\bfcode{getIdMatch}}{}{}
Retourne l'identifiant du match
\begin{quote}\begin{description}
\item[{Returns}] \leavevmode
l'identifiant du match

\item[{Return type}] \leavevmode
str

\end{description}\end{quote}

\end{fulllineitems}

\index{getMaCouleur() (Robot.Robot method)}

\begin{fulllineitems}
\phantomsection\label{index:Robot.Robot.getMaCouleur}\pysiglinewithargsret{\bfcode{getMaCouleur}}{}{}
Retourne la couleur du robot
\begin{quote}\begin{description}
\item[{Returns}] \leavevmode
la couleur du robot

\item[{Return type}] \leavevmode
int

\end{description}\end{quote}

\end{fulllineitems}

\index{getNbJoueurInitial() (Robot.Robot method)}

\begin{fulllineitems}
\phantomsection\label{index:Robot.Robot.getNbJoueurInitial}\pysiglinewithargsret{\bfcode{getNbJoueurInitial}}{}{}
Retourne le nombre de joueurs initial
\begin{quote}\begin{description}
\item[{Returns}] \leavevmode
le nombre de joueurs initial

\item[{Return type}] \leavevmode
int

\end{description}\end{quote}

\end{fulllineitems}

\index{getNbJoueurs() (Robot.Robot method)}

\begin{fulllineitems}
\phantomsection\label{index:Robot.Robot.getNbJoueurs}\pysiglinewithargsret{\bfcode{getNbJoueurs}}{}{}
Retourne le nombre de joueurs actuellement en train de jouer
\begin{quote}\begin{description}
\item[{Returns}] \leavevmode
le nombre de joueurs pouvant encore jouer

\item[{Return type}] \leavevmode
int

\end{description}\end{quote}

\end{fulllineitems}

\index{getStrategie() (Robot.Robot method)}

\begin{fulllineitems}
\phantomsection\label{index:Robot.Robot.getStrategie}\pysiglinewithargsret{\bfcode{getStrategie}}{}{}
Retourne la stratégie du robot
\begin{quote}\begin{description}
\item[{Returns}] \leavevmode
la stratégie du robot

\item[{Return type}] \leavevmode
Strategie

\end{description}\end{quote}

\end{fulllineitems}

\index{getTemps() (Robot.Robot method)}

\begin{fulllineitems}
\phantomsection\label{index:Robot.Robot.getTemps}\pysiglinewithargsret{\bfcode{getTemps}}{}{}
Retourne le temps actuel du jeu (fournie par le serveur)
\begin{quote}\begin{description}
\item[{Returns}] \leavevmode
le temps actuel du jeu

\item[{Return type}] \leavevmode
int

\end{description}\end{quote}

\end{fulllineitems}

\index{getTerrain() (Robot.Robot method)}

\begin{fulllineitems}
\phantomsection\label{index:Robot.Robot.getTerrain}\pysiglinewithargsret{\bfcode{getTerrain}}{}{}
Retourne le terrain du match en cours
\begin{quote}\begin{description}
\item[{Returns}] \leavevmode
le terrain du match en cours

\item[{Return type}] \leavevmode
Terrain

\end{description}\end{quote}

\end{fulllineitems}

\index{getUID() (Robot.Robot method)}

\begin{fulllineitems}
\phantomsection\label{index:Robot.Robot.getUID}\pysiglinewithargsret{\bfcode{getUID}}{}{}
Retourne l'identifiant du robot
\begin{quote}\begin{description}
\item[{Returns}] \leavevmode
l'identifiant du robot

\item[{Return type}] \leavevmode
str

\end{description}\end{quote}

\end{fulllineitems}

\index{getVitesse() (Robot.Robot method)}

\begin{fulllineitems}
\phantomsection\label{index:Robot.Robot.getVitesse}\pysiglinewithargsret{\bfcode{getVitesse}}{}{}
Retourne la vitesse du match en cours
\begin{quote}\begin{description}
\item[{Returns}] \leavevmode
la vitesse du match en cours

\item[{Return type}] \leavevmode
int

\end{description}\end{quote}

\end{fulllineitems}

\index{initialiserMatch() (Robot.Robot method)}

\begin{fulllineitems}
\phantomsection\label{index:Robot.Robot.initialiserMatch}\pysiglinewithargsret{\bfcode{initialiserMatch}}{\emph{init\_string}}{}
Initialise le robot pour un match.
A appeler dans la procédure `init\_pooo(init\_string)'

exemple : ``INIT20ac18ab-6d18-450e-94af-bee53fdc8fcaTO6{[}2{]};1;3CELLS:1(23,9)`2`30`8'I,2(41,55)`1`30`8'II,3(23,103)`1`20`5'I;2LINES:\href{mailto:1@3433OF2}{1@3433OF2},1@6502OF3''
\begin{quote}\begin{description}
\item[{Parameters}] \leavevmode
\textbf{init\_string} (\href{http://docs.python.org/library/functions.html\#str}{\emph{str}}) -- chaîne regroupant les informations envoyées par le serveur pour l'initialisation d'un nouveau match, sous la forme INIT.

\end{description}\end{quote}

\end{fulllineitems}

\index{partieEnCours() (Robot.Robot method)}

\begin{fulllineitems}
\phantomsection\label{index:Robot.Robot.partieEnCours}\pysiglinewithargsret{\bfcode{partieEnCours}}{}{}
Retourne vrai si une partie est en cours, faux sinon
\begin{quote}\begin{description}
\item[{Returns}] \leavevmode
vrai si une partie est en cours, faux sinon

\item[{Return type}] \leavevmode
booleen

\end{description}\end{quote}

\end{fulllineitems}

\index{peutJouer() (Robot.Robot method)}

\begin{fulllineitems}
\phantomsection\label{index:Robot.Robot.peutJouer}\pysiglinewithargsret{\bfcode{peutJouer}}{}{}
Retourne vrai si le robot peut jouer, et si il ne peut plus jouer
\begin{quote}\begin{description}
\item[{Returns}] \leavevmode
vrai si le robot a le droit de jouer, faux sinon

\item[{Return type}] \leavevmode
booleen

\end{description}\end{quote}

\end{fulllineitems}

\index{setStrategie() (Robot.Robot method)}

\begin{fulllineitems}
\phantomsection\label{index:Robot.Robot.setStrategie}\pysiglinewithargsret{\bfcode{setStrategie}}{\emph{strategie}}{}
Modifie la stratégie du robot
\begin{quote}\begin{description}
\item[{Parameters}] \leavevmode
\textbf{strategie} ({\hyperref[index:module-Strategie]{\emph{Strategie}}}) -- la nouvelle stratégie du robot

\end{description}\end{quote}

\end{fulllineitems}

\index{setTemps() (Robot.Robot method)}

\begin{fulllineitems}
\phantomsection\label{index:Robot.Robot.setTemps}\pysiglinewithargsret{\bfcode{setTemps}}{\emph{temps}}{}
Modifie le temps actuel du jeu, c'est à dire la variable temps du Robot, 
mais également touts les attributs temps\_actuel des mouvements (modifiant ainsi le temps restant).
\begin{quote}\begin{description}
\item[{Parameters}] \leavevmode
\textbf{temps} (\href{http://docs.python.org/library/functions.html\#int}{\emph{int}}) -- Le temps

\end{description}\end{quote}

\end{fulllineitems}

\index{updateTerrain() (Robot.Robot method)}

\begin{fulllineitems}
\phantomsection\label{index:Robot.Robot.updateTerrain}\pysiglinewithargsret{\bfcode{updateTerrain}}{\emph{state}}{}
Met à jour les informations sur le terrain en fonction de la chaîne passée en paramètre

exemple de chaine state
state = ``STATE20ac18ab-6d18-450e-94af-bee53fdc8fcaIS2;3CELLS:1{[}2{]}12`4,2{[}2{]}15`2,3{[}1{]}33`6;4MOVES:1\textless{}5{[}2{]}@232'\textgreater{}6{[}2{]}@488'\textgreater{}3{[}1{]}@4330`2,1\textless{}10{[}1{]}@2241`3''
\begin{quote}\begin{description}
\item[{Parameters}] \leavevmode
\textbf{state} (\href{http://docs.python.org/library/functions.html\#str}{\emph{str}}) -- la chaîne envoyée par le serveur, de la forme STATE

\end{description}\end{quote}

\end{fulllineitems}


\end{fulllineitems}



\chapter{Terrain}
\label{index:module-Terrain}\label{index:terrain}\index{Terrain (module)}\index{Terrain (class in Terrain)}

\begin{fulllineitems}
\phantomsection\label{index:Terrain.Terrain}\pysigline{\strong{class }\code{Terrain.}\bfcode{Terrain}}
Le terrain du jeu représente un graphe
\index{ajouterCellule() (Terrain.Terrain method)}

\begin{fulllineitems}
\phantomsection\label{index:Terrain.Terrain.ajouterCellule}\pysiglinewithargsret{\bfcode{ajouterCellule}}{\emph{cellule}}{}
Ajoute la cellule passée en paramètre dans le terrain
\begin{quote}\begin{description}
\item[{Parameters}] \leavevmode
\textbf{cellule} ({\hyperref[index:module-Cellule]{\emph{Cellule}}}) -- la cellule à ajouter

\end{description}\end{quote}

\end{fulllineitems}

\index{ajouterLien() (Terrain.Terrain method)}

\begin{fulllineitems}
\phantomsection\label{index:Terrain.Terrain.ajouterLien}\pysiglinewithargsret{\bfcode{ajouterLien}}{\emph{lien}}{}
Ajoute le lien passé en paramètre dans le terrain
\begin{quote}\begin{description}
\item[{Parameters}] \leavevmode
\textbf{lien} ({\hyperref[index:module-Lien]{\emph{Lien}}}) -- Le lien à ajouter

\item[{Raises TerrainException}] \leavevmode
si au moins l'une des cellule du lien n'est pas présente dans le terrain

\end{description}\end{quote}

\end{fulllineitems}

\index{dijkstra() (Terrain.Terrain method)}

\begin{fulllineitems}
\phantomsection\label{index:Terrain.Terrain.dijkstra}\pysiglinewithargsret{\bfcode{dijkstra}}{\emph{depart}, \emph{arrivee}}{}
Retourne le plus court chemin entre deux cellules passées en paramètre
\begin{quote}\begin{description}
\item[{Parameters}] \leavevmode\begin{itemize}
\item {} 
\textbf{depart} ({\hyperref[index:module-Cellule]{\emph{Cellule}}}) -- La cellule de départ

\item {} 
\textbf{arrivee} ({\hyperref[index:module-Cellule]{\emph{Cellule}}}) -- La cellule d'arrivée

\end{itemize}

\item[{Returns}] \leavevmode
le plus court chemin entre les deux cellules (une liste de numéro de cellule), et la distance totale à parcourir

\item[{Return type}] \leavevmode
( List\textless{}int\textgreater{} , int )

\end{description}\end{quote}

\end{fulllineitems}

\index{getCellule() (Terrain.Terrain method)}

\begin{fulllineitems}
\phantomsection\label{index:Terrain.Terrain.getCellule}\pysiglinewithargsret{\bfcode{getCellule}}{\emph{numero}}{}
Retourne la cellule du terrain ayant le numéro associé passé en paramètre
\begin{quote}\begin{description}
\item[{Parameters}] \leavevmode
\textbf{numero} (\href{http://docs.python.org/library/functions.html\#int}{\emph{int}}) -- Le numéro de la cellule recherchée

\item[{Returns}] \leavevmode
la cellule associée au numéro

\item[{Return type}] \leavevmode
:Cellule

\item[{Raises TerrainException}] \leavevmode
si il n'y a aucune cellule dans le terrain possédant ce numéro

\end{description}\end{quote}

\end{fulllineitems}

\index{getCellules() (Terrain.Terrain method)}

\begin{fulllineitems}
\phantomsection\label{index:Terrain.Terrain.getCellules}\pysiglinewithargsret{\bfcode{getCellules}}{}{}
Retourne le dictionnaire contenant la liste de toutes les cellules du terrain
\begin{quote}\begin{description}
\item[{Returns}] \leavevmode
le dictionnaire contenant les cellules du terrain

\item[{Return type}] \leavevmode
dict

\end{description}\end{quote}

\end{fulllineitems}

\index{getCellulesJoueur() (Terrain.Terrain method)}

\begin{fulllineitems}
\phantomsection\label{index:Terrain.Terrain.getCellulesJoueur}\pysiglinewithargsret{\bfcode{getCellulesJoueur}}{\emph{couleurJoueur}}{}
Retourne la liste des cellules appartenant au joueur ayant la couleur passée en paramètre (-1 pour le neutre).
\begin{quote}\begin{description}
\item[{Parameters}] \leavevmode
\textbf{couleurJoueur} (\href{http://docs.python.org/library/functions.html\#int}{\emph{int}}) -- la couleur du joueur

\item[{Returns}] \leavevmode
la liste des cellules de ce joueur

\item[{Return type}] \leavevmode
List\textless{}Cellule\textgreater{}

\end{description}\end{quote}

\end{fulllineitems}

\index{getCheminVersCellulePlusProche() (Terrain.Terrain method)}

\begin{fulllineitems}
\phantomsection\label{index:Terrain.Terrain.getCheminVersCellulePlusProche}\pysiglinewithargsret{\bfcode{getCheminVersCellulePlusProche}}{\emph{depart}, \emph{arrivees}}{}
Retourne le chemin depuis une cellule vers la cellule la plus proche selectionnée dans un ensemble
\begin{quote}\begin{description}
\item[{Parameters}] \leavevmode\begin{itemize}
\item {} 
\textbf{depart} ({\hyperref[index:module-Cellule]{\emph{Cellule}}}) -- La cellule de départ

\item {} 
\textbf{arrivees} (\emph{List\textless{}Cellules\textgreater{}}) -- L'ensemble des cellules d'arrivée

\end{itemize}

\item[{Returns}] \leavevmode
le chemin le plus court entre la cellule de départ et la cellule d'arrivée la plus proche (correspond à la liste des numéro des cellules composant le chemin)

\item[{Return type}] \leavevmode
List\textless{}int\textgreater{}

\end{description}\end{quote}

\end{fulllineitems}

\index{getComposantesConnexes() (Terrain.Terrain method)}

\begin{fulllineitems}
\phantomsection\label{index:Terrain.Terrain.getComposantesConnexes}\pysiglinewithargsret{\bfcode{getComposantesConnexes}}{}{}
Retourne la liste des composantes connexes du graphe (le terrain se comporte comme un graphe).
\begin{quote}\begin{description}
\item[{Returns}] \leavevmode
les composantes connexes du graphe.

\item[{Return type}] \leavevmode
List\textless{}Terrain\textgreater{}

\end{description}\end{quote}

\end{fulllineitems}

\index{getLien() (Terrain.Terrain method)}

\begin{fulllineitems}
\phantomsection\label{index:Terrain.Terrain.getLien}\pysiglinewithargsret{\bfcode{getLien}}{\emph{numeroLien}}{}
Retourne le lien du terrain ayant le numéro associé passé en paramètre
\begin{quote}\begin{description}
\item[{Parameters}] \leavevmode
\textbf{numeroLien} (\href{http://docs.python.org/library/functions.html\#int}{\emph{int}}) -- Le numéro de le lien recherché

\item[{Returns}] \leavevmode
le lien associée au numéro

\item[{Return type}] \leavevmode
Lien

\item[{Raises TerrainException}] \leavevmode
si il n'y a aucun Lien dans le terrain possédant ce numéro

\end{description}\end{quote}

\end{fulllineitems}

\index{getLienEntreCellules() (Terrain.Terrain method)}

\begin{fulllineitems}
\phantomsection\label{index:Terrain.Terrain.getLienEntreCellules}\pysiglinewithargsret{\bfcode{getLienEntreCellules}}{\emph{cellule\_1}, \emph{cellule\_2}}{}
Retourne le lien sur le terrain entre les 2 cellules passées en paramètre
\begin{quote}\begin{description}
\item[{Parameters}] \leavevmode\begin{itemize}
\item {} 
\textbf{cellule\_1} -- La première cellule

\item {} 
\textbf{cellule\_2} -- La deuxième cellule

\end{itemize}

\item[{Returns}] \leavevmode
le lien entre les deux cellules

\item[{Return type}] \leavevmode
Lien

\item[{Raises TerrainException}] \leavevmode
si le lien n'est pas présent dans le terrain

\end{description}\end{quote}

\end{fulllineitems}

\index{getLiens() (Terrain.Terrain method)}

\begin{fulllineitems}
\phantomsection\label{index:Terrain.Terrain.getLiens}\pysiglinewithargsret{\bfcode{getLiens}}{}{}
Retourne le dictionnaire contenant la liste de touts les liens du terrain
\begin{quote}\begin{description}
\item[{Returns}] \leavevmode
le dictionnaire contenant les liens du terrain

\item[{Return type}] \leavevmode
dict

\end{description}\end{quote}

\end{fulllineitems}

\index{getNbCellules() (Terrain.Terrain method)}

\begin{fulllineitems}
\phantomsection\label{index:Terrain.Terrain.getNbCellules}\pysiglinewithargsret{\bfcode{getNbCellules}}{}{}
Retourne le nombre de cellules présentes dans le terrain
\begin{quote}\begin{description}
\item[{Returns}] \leavevmode
le nombre de cellules sur le terrain

\item[{Return type}] \leavevmode
int

\end{description}\end{quote}

\end{fulllineitems}

\index{getNbLiens() (Terrain.Terrain method)}

\begin{fulllineitems}
\phantomsection\label{index:Terrain.Terrain.getNbLiens}\pysiglinewithargsret{\bfcode{getNbLiens}}{}{}
Retourne le nombre de liens présents dans le terrain
\begin{quote}\begin{description}
\item[{Returns}] \leavevmode
le nombre de liens sur le terrain

\item[{Return type}] \leavevmode
int

\end{description}\end{quote}

\end{fulllineitems}

\index{getSousGraphe() (Terrain.Terrain method)}

\begin{fulllineitems}
\phantomsection\label{index:Terrain.Terrain.getSousGraphe}\pysiglinewithargsret{\bfcode{getSousGraphe}}{\emph{listeCellules}}{}
Retourne le sous graphe contenant les cellules données en paramètre
\begin{quote}\begin{description}
\item[{Parameters}] \leavevmode
\textbf{listeCellules} (\emph{List\textless{}Cellules\textgreater{}}) -- La liste des cellules

\item[{Returns}] \leavevmode
le sous graphe correspondant

\item[{Return type}] \leavevmode
Terrain

\end{description}\end{quote}

\end{fulllineitems}

\index{getVoisinsCellule() (Terrain.Terrain method)}

\begin{fulllineitems}
\phantomsection\label{index:Terrain.Terrain.getVoisinsCellule}\pysiglinewithargsret{\bfcode{getVoisinsCellule}}{\emph{cellule}}{}
Retourne la liste des voisins de la cellule sur le terrain passée en paramètre
\begin{quote}\begin{description}
\item[{Parameters}] \leavevmode
\textbf{cellule} ({\hyperref[index:module-Cellule]{\emph{Cellule}}}) -- La cellule dont on veut récupérer la liste des voisins sur le terrain

\item[{Returns}] \leavevmode
la liste des cellules voisines à celle passée en paramètre

\item[{Return type}] \leavevmode
List\textless{}Cellule\textgreater{}

\item[{Raises TerrainException}] \leavevmode
si la cellule n'est pas présente sur le terrain

\end{description}\end{quote}

\end{fulllineitems}

\index{toString() (Terrain.Terrain method)}

\begin{fulllineitems}
\phantomsection\label{index:Terrain.Terrain.toString}\pysiglinewithargsret{\bfcode{toString}}{}{}
Retourne sous forme de chaine une représentation textuelle du terrain
\begin{quote}\begin{description}
\item[{Returns}] \leavevmode
le terrain sous forme de chaine de caractères

\item[{Return type}] \leavevmode
str

\end{description}\end{quote}

\end{fulllineitems}


\end{fulllineitems}



\chapter{Strategie}
\label{index:module-Strategie}\label{index:strategie}\index{Strategie (module)}\index{Strategie (class in Strategie)}

\begin{fulllineitems}
\phantomsection\label{index:Strategie.Strategie}\pysiglinewithargsret{\strong{class }\code{Strategie.}\bfcode{Strategie}}{\emph{robot}}{}
Classe abstraite à implémenter pour faire une stratégie
\begin{quote}\begin{description}
\item[{Parameters}] \leavevmode
\textbf{robot} ({\hyperref[index:module-Robot]{\emph{Robot}}}) -- Le robot devant prendre une decision

\end{description}\end{quote}
\index{decider() (Strategie.Strategie method)}

\begin{fulllineitems}
\phantomsection\label{index:Strategie.Strategie.decider}\pysiglinewithargsret{\bfcode{decider}}{}{}
Méthode abstraite, retourne la liste des mouvements à effectuer après analyse du terrain pour la prise de décision
\begin{quote}\begin{description}
\item[{Returns}] \leavevmode
la liste des nouveaux mouvements à effectuer

\item[{Return type}] \leavevmode
List\textless{}Mouvement\textgreater{}

\item[{Raises NotImplementedError}] \leavevmode
si la méthode n'a pas été redéfinie.

\end{description}\end{quote}

\end{fulllineitems}

\index{envoyerUnites() (Strategie.Strategie method)}

\begin{fulllineitems}
\phantomsection\label{index:Strategie.Strategie.envoyerUnites}\pysiglinewithargsret{\bfcode{envoyerUnites}}{\emph{depuis\_cellule}, \emph{vers\_cellules}, \emph{nb\_unites}}{}
Permet d'envoyer des unités attaquantes d'une cellule vers une autre, ne fait que modifier le terrain (ne les envoie pas au serveur).
On modifie le terrain afin de ne pas reprendre en compte les unitées déjà utilisées.
Retourne le mouvement créé !
\begin{quote}\begin{description}
\item[{Parameters}] \leavevmode\begin{itemize}
\item {} 
\textbf{depuis\_cellule} ({\hyperref[index:module-Cellule]{\emph{Cellule}}}) -- La cellule qui envoie les unités attaquantes

\item {} 
\textbf{vers\_cellules} ({\hyperref[index:module-Cellule]{\emph{Cellule}}}) -- La cellule qui reçoit les unités attaquantes

\end{itemize}

\item[{Returns}] \leavevmode
le mouvement créé

\item[{Return type}] \leavevmode
Mouvement

\end{description}\end{quote}

\end{fulllineitems}

\index{getRobot() (Strategie.Strategie method)}

\begin{fulllineitems}
\phantomsection\label{index:Strategie.Strategie.getRobot}\pysiglinewithargsret{\bfcode{getRobot}}{}{}
Retourne le robot
\begin{quote}\begin{description}
\item[{Returns}] \leavevmode
le robot

\item[{Return type}] \leavevmode
Robot

\end{description}\end{quote}

\end{fulllineitems}


\end{fulllineitems}



\chapter{StrategieAnalyse}
\label{index:module-StrategieAnalyse}\label{index:strategieanalyse}\index{StrategieAnalyse (module)}\index{StrategieAnalyse (class in StrategieAnalyse)}

\begin{fulllineitems}
\phantomsection\label{index:StrategieAnalyse.StrategieAnalyse}\pysiglinewithargsret{\strong{class }\code{StrategieAnalyse.}\bfcode{StrategieAnalyse}}{\emph{robot}}{}
Cette stratégie analyse le terrain afin de déduire où elle doit envoyer les unités des cellules.

L'envoi des unités ainsi que leur destination sera déterminée après analyse du terrain.
Les cellules seront réparties en deux groupes : les productrices et les attaquantes
\begin{quote}\begin{description}
\item[{Parameters}] \leavevmode
\textbf{robot} ({\hyperref[index:module-Robot]{\emph{Robot}}}) -- Le robot devant prendre une decision

\end{description}\end{quote}
\index{decider() (StrategieAnalyse.StrategieAnalyse method)}

\begin{fulllineitems}
\phantomsection\label{index:StrategieAnalyse.StrategieAnalyse.decider}\pysiglinewithargsret{\bfcode{decider}}{}{}
Retourne la liste des mouvements à effectuer après analyse du terrain pour la prise de décision.
\begin{quote}\begin{description}
\item[{Returns}] \leavevmode
la liste des nouveaux mouvements à effectuer

\item[{Return type}] \leavevmode
List\textless{}Mouvement\textgreater{}

\end{description}\end{quote}

\end{fulllineitems}

\index{determinerCible() (StrategieAnalyse.StrategieAnalyse method)}

\begin{fulllineitems}
\phantomsection\label{index:StrategieAnalyse.StrategieAnalyse.determinerCible}\pysiglinewithargsret{\bfcode{determinerCible}}{\emph{attaquante}}{}
Détermine la cible d'une cellule attaquante (en utilisant l'indice P sur tous les voisins ennemies de la cellule attaquante).
\begin{quote}\begin{description}
\item[{Parameters}] \leavevmode
\textbf{attaquante} ({\hyperref[index:module-Cellule]{\emph{Cellule}}}) -- la cellule attaquante cherchant une cible

\item[{Returns}] \leavevmode
la cellule cible

\item[{Return type}] \leavevmode
Cellule

\end{description}\end{quote}

\end{fulllineitems}

\index{envoyerUnitesAttaquantes() (StrategieAnalyse.StrategieAnalyse method)}

\begin{fulllineitems}
\phantomsection\label{index:StrategieAnalyse.StrategieAnalyse.envoyerUnitesAttaquantes}\pysiglinewithargsret{\bfcode{envoyerUnitesAttaquantes}}{\emph{composante}, \emph{mesCellules}}{}
Retourne la liste des mouvements correspondant aux mouvements des unités des cellules attaquantes vers les cellules ennemies les plus prometteuses.
\begin{quote}\begin{description}
\item[{Parameters}] \leavevmode\begin{itemize}
\item {} 
\textbf{composante} ({\hyperref[index:module-Terrain]{\emph{Terrain}}}) -- la partie du terrain (sous graphe) où se trouvent nos cellules

\item {} 
\textbf{mesCellules} (\emph{dict of list of Cellule}) -- dictionnaire de liste de cellules

\end{itemize}

\item[{Returns}] \leavevmode
la liste des mouvements des unités des cellules attaquantes vers les cellules ennemies les plus prometteuses.

\item[{Return type}] \leavevmode
List\textless{}Mouvement\textgreater{}

\end{description}\end{quote}

\end{fulllineitems}

\index{envoyerUnitesProductrices() (StrategieAnalyse.StrategieAnalyse method)}

\begin{fulllineitems}
\phantomsection\label{index:StrategieAnalyse.StrategieAnalyse.envoyerUnitesProductrices}\pysiglinewithargsret{\bfcode{envoyerUnitesProductrices}}{\emph{composante}, \emph{mesCellules}}{}
Retourne la liste des mouvements correspondant aux mouvements des unités des cellules productrices vers les cellules attaquantes les plus proches.
\begin{quote}\begin{description}
\item[{Parameters}] \leavevmode\begin{itemize}
\item {} 
\textbf{composante} ({\hyperref[index:module-Terrain]{\emph{Terrain}}}) -- la partie du terrain (sous graphe) où se trouvent nos cellules

\item {} 
\textbf{mesCellules} (\emph{dict of list of Cellule}) -- dictionnaire de liste de cellules

\end{itemize}

\item[{Returns}] \leavevmode
la liste des mouvements des unités des cellules productrices vers les cellules attaquantes les plus proches.

\item[{Return type}] \leavevmode
List\textless{}Mouvement\textgreater{}

\end{description}\end{quote}

\end{fulllineitems}

\index{getAttaquantesEnDanger() (StrategieAnalyse.StrategieAnalyse method)}

\begin{fulllineitems}
\phantomsection\label{index:StrategieAnalyse.StrategieAnalyse.getAttaquantesEnDanger}\pysiglinewithargsret{\bfcode{getAttaquantesEnDanger}}{\emph{attaquantes}}{}
Retourne la liste de nos cellules attaquantes en danger.
Une cellule attaquante est en danger lorsqu'il y a des unités ennemies se dirigeant vers elle.
\begin{quote}\begin{description}
\item[{Parameters}] \leavevmode
\textbf{attaquantes} (\emph{List\textless{}Cellule\textgreater{}}) -- la liste de toutes nos cellules attaquantes

\item[{Returns}] \leavevmode
nos cellules attaquantes en danger

\item[{Return type}] \leavevmode
List\textless{}Cellule\textgreater{}

\end{description}\end{quote}

\end{fulllineitems}

\index{getAttaquantesEnSurete() (StrategieAnalyse.StrategieAnalyse method)}

\begin{fulllineitems}
\phantomsection\label{index:StrategieAnalyse.StrategieAnalyse.getAttaquantesEnSurete}\pysiglinewithargsret{\bfcode{getAttaquantesEnSurete}}{\emph{attaquantes}, \emph{attaquantes\_en\_dangers}}{}
Retourne la liste de nos cellules attaquantes en sûreté
\begin{quote}\begin{description}
\item[{Return type}] \leavevmode
List\textless{}Cellule\textgreater{}

\end{description}\end{quote}

\end{fulllineitems}

\index{getCellulesAttaquantes() (StrategieAnalyse.StrategieAnalyse method)}

\begin{fulllineitems}
\phantomsection\label{index:StrategieAnalyse.StrategieAnalyse.getCellulesAttaquantes}\pysiglinewithargsret{\bfcode{getCellulesAttaquantes}}{\emph{mesCellules}}{}
Retourne la liste de nos cellules attaquantes.
Une cellule est attaquante si elle est reliée à au moins un ennemi.
\begin{quote}\begin{description}
\item[{Parameters}] \leavevmode
\textbf{mesCellules} (\emph{List\textless{}Cellule\textgreater{}}) -- nos cellules attaquantes

\item[{Return type}] \leavevmode
List\textless{}Cellule\textgreater{}

\end{description}\end{quote}

\end{fulllineitems}

\index{getCellulesProductrices() (StrategieAnalyse.StrategieAnalyse method)}

\begin{fulllineitems}
\phantomsection\label{index:StrategieAnalyse.StrategieAnalyse.getCellulesProductrices}\pysiglinewithargsret{\bfcode{getCellulesProductrices}}{\emph{mesCellules}}{}
Retourne la liste des cellules productrices.
Une cellule est productrice si elle n'est reliée à aucun ennemi.
\begin{quote}\begin{description}
\item[{Parameters}] \leavevmode
\textbf{mesCellules} (\emph{List\textless{}Cellule\textgreater{}}) -- nos cellules productrices

\item[{Return type}] \leavevmode
List\textless{}Cellule\textgreater{}

\end{description}\end{quote}

\end{fulllineitems}

\index{getCoutCellule() (StrategieAnalyse.StrategieAnalyse method)}

\begin{fulllineitems}
\phantomsection\label{index:StrategieAnalyse.StrategieAnalyse.getCoutCellule}\pysiglinewithargsret{\bfcode{getCoutCellule}}{\emph{cellule}}{}
Calcule le nombre d'unités que l'on doit envoyer sur une cellule ennemie afin de la capturer
Ce coût peut être négatif si la cellule nous appartient déjà.
\begin{quote}\begin{description}
\item[{Parameters}] \leavevmode
\textbf{cellule} ({\hyperref[index:module-Cellule]{\emph{Cellule}}}) -- Cellule ennemie

\item[{Returns}] \leavevmode
le nombre d'unités nécéssaire à la capture de la cellule

\item[{Return type}] \leavevmode
int

\end{description}\end{quote}

\end{fulllineitems}

\index{getFullProductrices() (StrategieAnalyse.StrategieAnalyse method)}

\begin{fulllineitems}
\phantomsection\label{index:StrategieAnalyse.StrategieAnalyse.getFullProductrices}\pysiglinewithargsret{\bfcode{getFullProductrices}}{\emph{productrices}, \emph{semi\_productrices}}{}
Retourne la liste de nos cellules entièrement productrices (full-productrices).
Une cellule est full-productrice si c'est une cellule productrice qui n'est reliée à aucune cellule attaquante.
\begin{quote}\begin{description}
\item[{Return type}] \leavevmode
List\textless{}Cellule\textgreater{}

\end{description}\end{quote}

\end{fulllineitems}

\index{getMesCellules() (StrategieAnalyse.StrategieAnalyse method)}

\begin{fulllineitems}
\phantomsection\label{index:StrategieAnalyse.StrategieAnalyse.getMesCellules}\pysiglinewithargsret{\bfcode{getMesCellules}}{}{}
Retourne la liste des cellules nous appartenant
\begin{quote}\begin{description}
\item[{Returns}] \leavevmode
les cellules nous appartenant

\item[{Return type}] \leavevmode
List\textless{}Cellule\textgreater{}

\end{description}\end{quote}

\end{fulllineitems}

\index{getSemiProductrices() (StrategieAnalyse.StrategieAnalyse method)}

\begin{fulllineitems}
\phantomsection\label{index:StrategieAnalyse.StrategieAnalyse.getSemiProductrices}\pysiglinewithargsret{\bfcode{getSemiProductrices}}{\emph{productrices}, \emph{attaquantes}}{}
Retourne la liste de nos cellules semi-productrices.
Une cellule est semi-productrice si c'est une cellule productrice reliée à au moins une cellule attaquante
\begin{quote}\begin{description}
\item[{Returns}] \leavevmode
nos cellules semi-productrices

\item[{Return type}] \leavevmode
List\textless{}Cellule\textgreater{}

\end{description}\end{quote}

\end{fulllineitems}

\index{indiceP() (StrategieAnalyse.StrategieAnalyse method)}

\begin{fulllineitems}
\phantomsection\label{index:StrategieAnalyse.StrategieAnalyse.indiceP}\pysiglinewithargsret{\bfcode{indiceP}}{\emph{origine}, \emph{cellule}}{}
Calcule l'indice P d'une cellule par rapport à la cellule d'origine voulant envoyer ses unités.
\begin{quote}\begin{description}
\item[{Parameters}] \leavevmode\begin{itemize}
\item {} 
\textbf{origine} ({\hyperref[index:module-Cellule]{\emph{Cellule}}}) -- La cellule d'origine

\item {} 
\textbf{cellule} ({\hyperref[index:module-Cellule]{\emph{Cellule}}}) -- La cellule dont on veut calculer l'indice P

\end{itemize}

\item[{Returns}] \leavevmode
l'indice P de la cellule cible

\item[{Return type}] \leavevmode
float

\end{description}\end{quote}

\end{fulllineitems}

\index{nbUnitesAEnvoyer() (StrategieAnalyse.StrategieAnalyse method)}

\begin{fulllineitems}
\phantomsection\label{index:StrategieAnalyse.StrategieAnalyse.nbUnitesAEnvoyer}\pysiglinewithargsret{\bfcode{nbUnitesAEnvoyer}}{\emph{attaquante}, \emph{cellule\_cible}}{}
Détermine le nombre d'unités à envoyer de la cellule attaquante vers une cellule\_cible en fonction de l'excédent de l'attaquant et du coût de la cible. 
Peut renvoyer 0, dans ce cas la, la cellule attaquante ne devra pas envoyer d'unités.
\begin{quote}\begin{description}
\item[{Parameters}] \leavevmode\begin{itemize}
\item {} 
\textbf{attaquante} ({\hyperref[index:module-Cellule]{\emph{Cellule}}}) -- la cellule de départ voulant déterminer le nombre d'unités à envoyer

\item {} 
\textbf{cellule\_cible} ({\hyperref[index:module-Cellule]{\emph{Cellule}}}) -- la cellule ciblée

\end{itemize}

\item[{Returns}] \leavevmode
le nombre d'unités à envoyer vers la cellule ciblée

\item[{Return type}] \leavevmode
int

\end{description}\end{quote}

\end{fulllineitems}


\end{fulllineitems}



\chapter{StrategiePrevision}
\label{index:module-StrategiePrevision}\label{index:strategieprevision}\index{StrategiePrevision (module)}\index{StrategiePrevision (class in StrategiePrevision)}

\begin{fulllineitems}
\phantomsection\label{index:StrategiePrevision.StrategiePrevision}\pysiglinewithargsret{\strong{class }\code{StrategiePrevision.}\bfcode{StrategiePrevision}}{\emph{robot}}{}
Cette stratégie est une surcouche de la stratégie Analyse.
Elle essaie de prendre les cellules au bon moment, afin de subir le moins de pertes possible
\begin{quote}\begin{description}
\item[{Parameters}] \leavevmode
\textbf{robot} ({\hyperref[index:module-Robot]{\emph{Robot}}}) -- Le robot devant prendre une decision

\end{description}\end{quote}
\index{determinerCible() (StrategiePrevision.StrategiePrevision method)}

\begin{fulllineitems}
\phantomsection\label{index:StrategiePrevision.StrategiePrevision.determinerCible}\pysiglinewithargsret{\bfcode{determinerCible}}{\emph{attaquante}}{}
Dértermine la cible d'une cellule attaquante
\begin{quote}\begin{description}
\item[{Parameters}] \leavevmode
\textbf{attaquante} ({\hyperref[index:module-Cellule]{\emph{Cellule}}}) -- la cellule attaquante cherchant une cible

\item[{Returns}] \leavevmode
la cellule cible

\item[{Return type}] \leavevmode
Cellule

\end{description}\end{quote}

\end{fulllineitems}


\end{fulllineitems}



\chapter{StrategieAleatoire}
\label{index:module-StrategieAleatoire}\label{index:strategiealeatoire}\index{StrategieAleatoire (module)}\index{StrategieAleatoire (class in StrategieAleatoire)}

\begin{fulllineitems}
\phantomsection\label{index:StrategieAleatoire.StrategieAleatoire}\pysiglinewithargsret{\strong{class }\code{StrategieAleatoire.}\bfcode{StrategieAleatoire}}{\emph{robot}}{}
1ere stratégie qui attaque aléatoirement.

L'envoi des unités ainsi que leur destination sera déterminée aléatoirement.
\begin{quote}\begin{description}
\item[{Parameters}] \leavevmode
\textbf{robot} ({\hyperref[index:module-Robot]{\emph{Robot}}}) -- Le robot devant prendre une decision

\end{description}\end{quote}
\index{decider() (StrategieAleatoire.StrategieAleatoire method)}

\begin{fulllineitems}
\phantomsection\label{index:StrategieAleatoire.StrategieAleatoire.decider}\pysiglinewithargsret{\bfcode{decider}}{}{}
Retourne la liste des mouvements à effectuer après analyse du terrain pour la prise de décision.
\begin{quote}\begin{description}
\item[{Returns}] \leavevmode
la liste des nouveaux mouvements à effectuer

\item[{Return type}] \leavevmode
List\textless{}Mouvement\textgreater{}

\end{description}\end{quote}

\end{fulllineitems}


\end{fulllineitems}



\chapter{Graphique}
\label{index:module-Graphique}\label{index:graphique}\index{Graphique (module)}\index{Graphique (class in Graphique)}

\begin{fulllineitems}
\phantomsection\label{index:Graphique.Graphique}\pysiglinewithargsret{\strong{class }\code{Graphique.}\bfcode{Graphique}}{\emph{robot}}{}
Représente L'interface graphique du jeu.
\begin{quote}\begin{description}
\item[{Parameters}] \leavevmode
\textbf{robot} ({\hyperref[index:module-Robot]{\emph{Robot}}}) -- Le robot du jeu

\end{description}\end{quote}
\index{create\_circle() (Graphique.Graphique method)}

\begin{fulllineitems}
\phantomsection\label{index:Graphique.Graphique.create_circle}\pysiglinewithargsret{\bfcode{create\_circle}}{\emph{x}, \emph{y}, \emph{r}, \emph{**kwargs}}{}
Créer un cercle dans le canvas
\begin{quote}\begin{description}
\item[{Parameters}] \leavevmode\begin{itemize}
\item {} 
\textbf{x} (\href{http://docs.python.org/library/functions.html\#int}{\emph{int}}) -- La position en X du centre du cercle

\item {} 
\textbf{y} (\href{http://docs.python.org/library/functions.html\#int}{\emph{int}}) -- La position en Y du centre du cercle

\item {} 
\textbf{r} (\href{http://docs.python.org/library/functions.html\#int}{\emph{int}}) -- Le rayon du cercle

\item {} 
\textbf{**kwargs} -- 
Paramètres facultatifs de tkinter pour créer un cercle


\end{itemize}

\end{description}\end{quote}

\end{fulllineitems}

\index{decalage\_centre\_cercle() (Graphique.Graphique method)}

\begin{fulllineitems}
\phantomsection\label{index:Graphique.Graphique.decalage_centre_cercle}\pysiglinewithargsret{\bfcode{decalage\_centre\_cercle}}{\emph{u}, \emph{v}}{}
Permet de récupérer le point d'intersection entre le bord du cercle et le lien.
\begin{quote}\begin{description}
\item[{Parameters}] \leavevmode\begin{itemize}
\item {} 
\textbf{u} ({\hyperref[index:module-Cellule]{\emph{Cellule}}}) -- La cellule de départ

\item {} 
\textbf{v} ({\hyperref[index:module-Cellule]{\emph{Cellule}}}) -- La cellule d'arrivée

\end{itemize}

\end{description}\end{quote}

\end{fulllineitems}

\index{dessinerCellule() (Graphique.Graphique method)}

\begin{fulllineitems}
\phantomsection\label{index:Graphique.Graphique.dessinerCellule}\pysiglinewithargsret{\bfcode{dessinerCellule}}{\emph{cellule}}{}
Permet de dessiner une cellule dans la fenêtre
\begin{quote}\begin{description}
\item[{Parameters}] \leavevmode
\textbf{cellule} ({\hyperref[index:module-Cellule]{\emph{Cellule}}}) -- La cellule à dessiner

\end{description}\end{quote}

\end{fulllineitems}

\index{dessinerCellules() (Graphique.Graphique method)}

\begin{fulllineitems}
\phantomsection\label{index:Graphique.Graphique.dessinerCellules}\pysiglinewithargsret{\bfcode{dessinerCellules}}{}{}
Utilise la méthode ``dessinerCellule'' pour dessiner l'ensemble des cellules du jeu dans la fenêtre

\end{fulllineitems}

\index{dessinerLien() (Graphique.Graphique method)}

\begin{fulllineitems}
\phantomsection\label{index:Graphique.Graphique.dessinerLien}\pysiglinewithargsret{\bfcode{dessinerLien}}{\emph{lien}}{}
Permet de dessiner un lien dans la fenêtre.
\begin{quote}\begin{description}
\item[{Parameters}] \leavevmode
\textbf{lien} ({\hyperref[index:module-Lien]{\emph{Lien}}}) -- Le lien à dessiner

\end{description}\end{quote}

\end{fulllineitems}

\index{dessinerLiens() (Graphique.Graphique method)}

\begin{fulllineitems}
\phantomsection\label{index:Graphique.Graphique.dessinerLiens}\pysiglinewithargsret{\bfcode{dessinerLiens}}{}{}
Utilise la méthode ``dessinerLien'' pour dessiner l'ensemble des liens sur la fenêtre

\end{fulllineitems}

\index{dessinerMouvement() (Graphique.Graphique method)}

\begin{fulllineitems}
\phantomsection\label{index:Graphique.Graphique.dessinerMouvement}\pysiglinewithargsret{\bfcode{dessinerMouvement}}{\emph{mouvement}}{}
Permet de dessiner un mouvement se déplaçant sur un lien
\begin{quote}\begin{description}
\item[{Parameters}] \leavevmode
\textbf{mouvement} ({\hyperref[index:module-Mouvement]{\emph{Mouvement}}}) -- le mouvement à dessiner

\end{description}\end{quote}

\end{fulllineitems}

\index{dessinerMouvements() (Graphique.Graphique method)}

\begin{fulllineitems}
\phantomsection\label{index:Graphique.Graphique.dessinerMouvements}\pysiglinewithargsret{\bfcode{dessinerMouvements}}{}{}
Utilise la méthode ``dessinerMouvement'' pour dessiner tous les mouvements en cours du terrain

\end{fulllineitems}

\index{getTrueCoordonneeCellule() (Graphique.Graphique method)}

\begin{fulllineitems}
\phantomsection\label{index:Graphique.Graphique.getTrueCoordonneeCellule}\pysiglinewithargsret{\bfcode{getTrueCoordonneeCellule}}{\emph{cellule}}{}
Permet d'adapter les coordonnées de la cellule envoyées par le serveur en fonction des dimensions de la fenêtre
\begin{quote}\begin{description}
\item[{Parameters}] \leavevmode
\textbf{cellule} ({\hyperref[index:module-Cellule]{\emph{Cellule}}}) -- La cellule à adapter

\item[{Return type}] \leavevmode
float

\end{description}\end{quote}

\end{fulllineitems}

\index{getTrueRayonCellule() (Graphique.Graphique method)}

\begin{fulllineitems}
\phantomsection\label{index:Graphique.Graphique.getTrueRayonCellule}\pysiglinewithargsret{\bfcode{getTrueRayonCellule}}{\emph{cellule}}{}
Permet d'adapter le rayon de la cellule envoyé par le serveur en fonction des dimensions de la fenêtre
\begin{quote}\begin{description}
\item[{Parameters}] \leavevmode
\textbf{cellule} ({\hyperref[index:module-Cellule]{\emph{Cellule}}}) -- La cellule à adapter

\item[{Return type}] \leavevmode
float

\end{description}\end{quote}

\end{fulllineitems}

\index{redessinerCellule() (Graphique.Graphique method)}

\begin{fulllineitems}
\phantomsection\label{index:Graphique.Graphique.redessinerCellule}\pysiglinewithargsret{\bfcode{redessinerCellule}}{\emph{cellule}}{}
Permet de redessiner une cellule
\begin{quote}\begin{description}
\item[{Parameters}] \leavevmode
\textbf{cellule} ({\hyperref[index:module-Cellule]{\emph{Cellule}}}) -- La cellule à redessiner

\end{description}\end{quote}

\end{fulllineitems}

\index{redessinerCellules() (Graphique.Graphique method)}

\begin{fulllineitems}
\phantomsection\label{index:Graphique.Graphique.redessinerCellules}\pysiglinewithargsret{\bfcode{redessinerCellules}}{}{}
Utilise la méthode ``redessinerCellule'' pour redessiner toutes les cellules suivant l'évolution du jeu.

\end{fulllineitems}


\end{fulllineitems}

\index{Point (class in Graphique)}

\begin{fulllineitems}
\phantomsection\label{index:Graphique.Point}\pysiglinewithargsret{\strong{class }\code{Graphique.}\bfcode{Point}}{\emph{x}, \emph{y}}{}
Permet de créer un point aux coordonnées spécifiées en paramètre.

\end{fulllineitems}

\index{Vecteur (class in Graphique)}

\begin{fulllineitems}
\phantomsection\label{index:Graphique.Vecteur}\pysiglinewithargsret{\strong{class }\code{Graphique.}\bfcode{Vecteur}}{\emph{a}, \emph{b}}{}
Permet de représenter un vecteur avec les coordonnées spécifiées en paramètre.
\index{norme() (Graphique.Vecteur method)}

\begin{fulllineitems}
\phantomsection\label{index:Graphique.Vecteur.norme}\pysiglinewithargsret{\bfcode{norme}}{}{}
Calcule la norme du vecteur
\begin{quote}\begin{description}
\item[{Return type}] \leavevmode
float

\end{description}\end{quote}

\end{fulllineitems}

\index{produitScalaireBAC() (Graphique.Vecteur method)}

\begin{fulllineitems}
\phantomsection\label{index:Graphique.Vecteur.produitScalaireBAC}\pysiglinewithargsret{\bfcode{produitScalaireBAC}}{\emph{ab}, \emph{ac}}{}
Retourne le produit scalaire des deux vecteurs passés en paramètre
\begin{quote}\begin{description}
\item[{Parameters}] \leavevmode\begin{itemize}
\item {} 
\textbf{ab} (\emph{Vecteur}) -- Le premier vecteur

\item {} 
\textbf{ac} (\emph{Vecteur}) -- Le second vecteur

\end{itemize}

\item[{Return type}] \leavevmode
float

\end{description}\end{quote}

\end{fulllineitems}

\index{rotation() (Graphique.Vecteur method)}

\begin{fulllineitems}
\phantomsection\label{index:Graphique.Vecteur.rotation}\pysiglinewithargsret{\bfcode{rotation}}{\emph{angle}}{}
Rotation d'un vecteur de l'angle passé en paramètre
\begin{quote}\begin{description}
\item[{Parameters}] \leavevmode
\textbf{angle} (\href{http://docs.python.org/library/functions.html\#float}{\emph{float}}) -- l'angle de rotation à appliquer au vecteur (en radians)

\end{description}\end{quote}

\end{fulllineitems}

\index{translationPoint() (Graphique.Vecteur method)}

\begin{fulllineitems}
\phantomsection\label{index:Graphique.Vecteur.translationPoint}\pysiglinewithargsret{\bfcode{translationPoint}}{\emph{point}}{}
Effectue une translation du point passé en paramètre par rapport au vecteur
\begin{quote}\begin{description}
\item[{Parameters}] \leavevmode
\textbf{point} (\emph{Point}) -- Le point à translater

\item[{Return type}] \leavevmode
Point

\end{description}\end{quote}

\end{fulllineitems}


\end{fulllineitems}



\chapter{Exceptions}
\label{index:module-Exceptions}\label{index:exceptions}\index{Exceptions (module)}\index{CelluleException}

\begin{fulllineitems}
\phantomsection\label{index:Exceptions.CelluleException}\pysiglinewithargsret{\strong{exception }\code{Exceptions.}\bfcode{CelluleException}}{\emph{message}}{}
Se déclenche lors d'une exception dans la classe Cellule

\end{fulllineitems}

\index{LienException}

\begin{fulllineitems}
\phantomsection\label{index:Exceptions.LienException}\pysigline{\strong{exception }\code{Exceptions.}\bfcode{LienException}}
Se déclenche lors d'une exception dans la classe Lien

\end{fulllineitems}

\index{MouvementException}

\begin{fulllineitems}
\phantomsection\label{index:Exceptions.MouvementException}\pysigline{\strong{exception }\code{Exceptions.}\bfcode{MouvementException}}
Se déclenche lors d'une exception dans la classe Mouvement

\end{fulllineitems}

\index{RobotException}

\begin{fulllineitems}
\phantomsection\label{index:Exceptions.RobotException}\pysigline{\strong{exception }\code{Exceptions.}\bfcode{RobotException}}
Se déclenche lors d'une exception dans la classe Robot

\end{fulllineitems}

\index{TerrainException}

\begin{fulllineitems}
\phantomsection\label{index:Exceptions.TerrainException}\pysigline{\strong{exception }\code{Exceptions.}\bfcode{TerrainException}}
Se déclenche lors d'une exception dans la classe Terrain

\end{fulllineitems}



\chapter{Graphique\_lollipooo}
\label{index:graphique-lollipooo}\index{register\_pooo() (in module graphique\_lolipooo)}

\begin{fulllineitems}
\phantomsection\label{index:graphique_lolipooo.register_pooo}\pysiglinewithargsret{\code{graphique\_lolipooo.}\bfcode{register\_pooo}}{\emph{uid}}{}
Inscrit un joueur et initialise le robot pour la compétition
\begin{quote}\begin{description}
\item[{Parameters}] \leavevmode
\textbf{uid} (\emph{chaîne de caractères str(UUID)}) -- identifiant utilisateur

\item[{Example}] \leavevmode
\end{description}\end{quote}

``0947e717-02a1-4d83-9470-a941b6e8ed07''

\end{fulllineitems}

\index{init\_pooo() (in module graphique\_lolipooo)}

\begin{fulllineitems}
\phantomsection\label{index:graphique_lolipooo.init_pooo}\pysiglinewithargsret{\code{graphique\_lolipooo.}\bfcode{init\_pooo}}{\emph{init\_string}}{}
Initialise le robot pour un match
\begin{quote}\begin{description}
\item[{Parameters}] \leavevmode
\textbf{init\_string} (\emph{chaîne de caractères (utf-8 string)}) -- instruction du protocole de communication de Pooo (voire ci-dessous)

\end{description}\end{quote}

INIT\textless{}matchid\textgreater{}TO\textless{}\#players\textgreater{}{[}\textless{}me\textgreater{}{]};\textless{}speed\textgreater{};    \textless{}\#cells\textgreater{}CELLS:\textless{}cellid\textgreater{}(\textless{}x\textgreater{},\textless{}y\textgreater{})'\textless{}radius\textgreater{}'\textless{}offsize\textgreater{}'\textless{}defsize\textgreater{}'\textless{}prod\textgreater{},...;    \textless{}\#lines\textgreater{}LINES:\textless{}cellid\textgreater{}@\textless{}dist\textgreater{}OF\textless{}cellid\textgreater{},...

\textless{}me\textgreater{} et \textless{}owner\textgreater{} désignent des numéros de `couleur' attribués aux joueurs. La couleur 0 est le neutre.
le neutre n'est pas compté dans l'effectif de joueurs (\textless{}\#players\textgreater{}).
`...' signifie que l'on répète la séquence précédente autant de fois qu'il y a de cellules (ou d'arêtes).
0CELLS ou 0LINES sont des cas particuliers sans suffixe.
\textless{}dist\textgreater{} est la distance qui sépare 2 cellules, exprimée en... millisecondes !
/!attention: un match à vitesse x2 réduit de moitié le temps effectif de trajet d'une cellule à l'autre par rapport à l'indication \textless{}dist\textgreater{}.
De manière générale temps\_de\_trajet=\textless{}dist\textgreater{}/vitesse (division entière).
\begin{quote}\begin{description}
\item[{Example}] \leavevmode
\end{description}\end{quote}

``INIT20ac18ab-6d18-450e-94af-bee53fdc8fcaTO6{[}2{]};1;3CELLS:1(23,9)`2`30`8'I,2(41,55)`1`30`8'II,3(23,103)`1`20`5'I;2LINES:\href{mailto:1@3433OF2}{1@3433OF2},1@6502OF3''

\end{fulllineitems}

\index{play\_pooo() (in module graphique\_lolipooo)}

\begin{fulllineitems}
\phantomsection\label{index:graphique_lolipooo.play_pooo}\pysiglinewithargsret{\code{graphique\_lolipooo.}\bfcode{play\_pooo}}{}{}
Active le robot-joueur

\end{fulllineitems}

\index{updateGraphique() (in module graphique\_lolipooo)}

\begin{fulllineitems}
\phantomsection\label{index:graphique_lolipooo.updateGraphique}\pysiglinewithargsret{\code{graphique\_lolipooo.}\bfcode{updateGraphique}}{\emph{graphique}}{}
Permet la mise à jour automatique de l'interface graphique
\begin{quote}\begin{description}
\item[{Parameters}] \leavevmode
\textbf{graphique} ({\hyperref[index:module-Graphique]{\emph{Graphique}}}) -- l'interface graphique

\end{description}\end{quote}

\end{fulllineitems}

\index{updateTime() (in module graphique\_lolipooo)}

\begin{fulllineitems}
\phantomsection\label{index:graphique_lolipooo.updateTime}\pysiglinewithargsret{\code{graphique\_lolipooo.}\bfcode{updateTime}}{\emph{robot}}{}
Permet la mise à jour automatique du temps.
\begin{quote}\begin{description}
\item[{Parameters}] \leavevmode
\textbf{robot} ({\hyperref[index:module-Robot]{\emph{Robot}}}) -- le robot

\end{description}\end{quote}

\end{fulllineitems}

\index{updateGame() (in module graphique\_lolipooo)}

\begin{fulllineitems}
\phantomsection\label{index:graphique_lolipooo.updateGame}\pysiglinewithargsret{\code{graphique\_lolipooo.}\bfcode{updateGame}}{\emph{robot}}{}
Permet la mise à jour automatique de l'état du jeu (terrain...).
\begin{quote}\begin{description}
\item[{Parameters}] \leavevmode
\textbf{robot} ({\hyperref[index:module-Robot]{\emph{Robot}}}) -- le robot

\end{description}\end{quote}

\end{fulllineitems}

\index{sendDecisions() (in module graphique\_lolipooo)}

\begin{fulllineitems}
\phantomsection\label{index:graphique_lolipooo.sendDecisions}\pysiglinewithargsret{\code{graphique\_lolipooo.}\bfcode{sendDecisions}}{\emph{robot}}{}
Permet l'envoi automatique des decisions au serveur.
\begin{quote}\begin{description}
\item[{Parameters}] \leavevmode
\textbf{robot} ({\hyperref[index:module-Robot]{\emph{Robot}}}) -- le robot

\end{description}\end{quote}

\end{fulllineitems}



\chapter{lollipooo}
\label{index:lollipooo}\index{register\_pooo() (in module lolipooo)}

\begin{fulllineitems}
\phantomsection\label{index:lolipooo.register_pooo}\pysiglinewithargsret{\code{lolipooo.}\bfcode{register\_pooo}}{\emph{uid}}{}
Inscrit un joueur et initialise le robot pour la compétition
\begin{quote}\begin{description}
\item[{Parameters}] \leavevmode
\textbf{uid} (\emph{chaîne de caractères str(UUID)}) -- identifiant utilisateur

\item[{Example}] \leavevmode
\end{description}\end{quote}

``0947e717-02a1-4d83-9470-a941b6e8ed07''

\end{fulllineitems}

\index{init\_pooo() (in module lolipooo)}

\begin{fulllineitems}
\phantomsection\label{index:lolipooo.init_pooo}\pysiglinewithargsret{\code{lolipooo.}\bfcode{init\_pooo}}{\emph{init\_string}}{}
Initialise le robot pour un match
\begin{quote}
\begin{quote}\begin{description}
\item[{param init\_string}] \leavevmode
instruction du protocole de communication de Pooo (voire ci-dessous)

\item[{type init\_string}] \leavevmode
chaîne de caractères (utf-8 string)

\end{description}\end{quote}
\end{quote}

INIT\textless{}matchid\textgreater{}TO\textless{}\#players\textgreater{}{[}\textless{}me\textgreater{}{]};\textless{}speed\textgreater{};       \textless{}\#cells\textgreater{}CELLS:\textless{}cellid\textgreater{}(\textless{}x\textgreater{},\textless{}y\textgreater{})'\textless{}radius\textgreater{}'\textless{}offsize\textgreater{}'\textless{}defsize\textgreater{}'\textless{}prod\textgreater{},...;       \textless{}\#lines\textgreater{}LINES:\textless{}cellid\textgreater{}@\textless{}dist\textgreater{}OF\textless{}cellid\textgreater{},...

\textless{}me\textgreater{} et \textless{}owner\textgreater{} désignent des numéros de `couleur' attribués aux joueurs. La couleur 0 est le neutre.
le neutre n'est pas compté dans l'effectif de joueurs (\textless{}\#players\textgreater{}).
`...' signifie que l'on répète la séquence précédente autant de fois qu'il y a de cellules (ou d'arêtes).
0CELLS ou 0LINES sont des cas particuliers sans suffixe.
\textless{}dist\textgreater{} est la distance qui sépare 2 cellules, exprimée en... millisecondes !
/!attention: un match à vitesse x2 réduit de moitié le temps effectif de trajet d'une cellule à l'autre par rapport à l'indication \textless{}dist\textgreater{}.
De manière générale temps\_de\_trajet=\textless{}dist\textgreater{}/vitesse (division entière).
\begin{quote}
\begin{quote}\begin{description}
\item[{Example}] \leavevmode
\end{description}\end{quote}

``INIT20ac18ab-6d18-450e-94af-bee53fdc8fcaTO6{[}2{]};1;3CELLS:1(23,9)`2`30`8'I,2(41,55)`1`30`8'II,3(23,103)`1`20`5'I;2LINES:\href{mailto:1@3433OF2}{1@3433OF2},1@6502OF3''
\end{quote}

\end{fulllineitems}

\index{play\_pooo() (in module lolipooo)}

\begin{fulllineitems}
\phantomsection\label{index:lolipooo.play_pooo}\pysiglinewithargsret{\code{lolipooo.}\bfcode{play\_pooo}}{}{}
Active le robot-joueur

\end{fulllineitems}

\index{updateGraphique() (in module lolipooo)}

\begin{fulllineitems}
\phantomsection\label{index:lolipooo.updateGraphique}\pysiglinewithargsret{\code{lolipooo.}\bfcode{updateGraphique}}{\emph{graphique}}{}
Permet la mise à jour automatique de l'interface graphique
\begin{quote}\begin{description}
\item[{Parameters}] \leavevmode
\textbf{graphique} ({\hyperref[index:module-Graphique]{\emph{Graphique}}}) -- l'interface graphique

\end{description}\end{quote}

\end{fulllineitems}

\index{updateTime() (in module lolipooo)}

\begin{fulllineitems}
\phantomsection\label{index:lolipooo.updateTime}\pysiglinewithargsret{\code{lolipooo.}\bfcode{updateTime}}{\emph{robot}}{}
Permet la mise à jour automatique du temps.
\begin{quote}\begin{description}
\item[{Parameters}] \leavevmode
\textbf{robot} ({\hyperref[index:module-Robot]{\emph{Robot}}}) -- le robot

\end{description}\end{quote}

\end{fulllineitems}

\index{updateGame() (in module lolipooo)}

\begin{fulllineitems}
\phantomsection\label{index:lolipooo.updateGame}\pysiglinewithargsret{\code{lolipooo.}\bfcode{updateGame}}{\emph{robot}}{}
Permet la mise à jour automatique de l'état du jeu (terrain...).
\begin{quote}\begin{description}
\item[{Parameters}] \leavevmode
\textbf{robot} ({\hyperref[index:module-Robot]{\emph{Robot}}}) -- le robot

\end{description}\end{quote}

\end{fulllineitems}

\index{sendDecisions() (in module lolipooo)}

\begin{fulllineitems}
\phantomsection\label{index:lolipooo.sendDecisions}\pysiglinewithargsret{\code{lolipooo.}\bfcode{sendDecisions}}{\emph{robot}}{}
Permet l'envoi automatique des decisions au serveur.
\begin{quote}\begin{description}
\item[{Parameters}] \leavevmode
\textbf{robot} ({\hyperref[index:module-Robot]{\emph{Robot}}}) -- le robot

\end{description}\end{quote}

\end{fulllineitems}



\renewcommand{\indexname}{Python Module Index}
\begin{theindex}
\def\bigletter#1{{\Large\sffamily#1}\nopagebreak\vspace{1mm}}
\bigletter{c}
\item {\texttt{Cellule}}, \pageref{index:module-Cellule}
\indexspace
\bigletter{e}
\item {\texttt{Exceptions}}, \pageref{index:module-Exceptions}
\indexspace
\bigletter{g}
\item {\texttt{Graphique}}, \pageref{index:module-Graphique}
\indexspace
\bigletter{l}
\item {\texttt{Lien}}, \pageref{index:module-Lien}
\indexspace
\bigletter{m}
\item {\texttt{Mouvement}}, \pageref{index:module-Mouvement}
\indexspace
\bigletter{r}
\item {\texttt{Robot}}, \pageref{index:module-Robot}
\indexspace
\bigletter{s}
\item {\texttt{Strategie}}, \pageref{index:module-Strategie}
\item {\texttt{StrategieAleatoire}}, \pageref{index:module-StrategieAleatoire}
\item {\texttt{StrategieAnalyse}}, \pageref{index:module-StrategieAnalyse}
\item {\texttt{StrategiePrevision}}, \pageref{index:module-StrategiePrevision}
\indexspace
\bigletter{t}
\item {\texttt{Terrain}}, \pageref{index:module-Terrain}
\end{theindex}

\renewcommand{\indexname}{Index}
\printindex
\end{document}
